% Token Issuance
\section{Выпуск токенов (ВТ)}\label{TI}

\subsection{Обзор}\label{TI.Intro}

Пакет ВТ устанавливает требования по выбору типов токенов и их комбинаций, 
надежности токенов для соответствия тому или иному уровню гарантий 
аутентификации. 

Все рассматриваемые в настоящем стандарте токены содержат секреты
аутентификации: статический пароль, личный или секретный ключ, другое.
%
Статический пароль способен запомнить человек, и поэтому он относится к фактору
<<что я знаю>>. Секретные и личные ключи пользователь запомнить и обработать не
может, они не существуют автономно, а являются частью программного или
аппаратного токена. Этот токен относится к фактору <<чем я владею>>.
%
Биометрические токены, основанные на биометрических данных и относящиеся к 
фактору <<кто я>>, не содержат приемлемых секретов аутентификации. В настоящем 
стандарте биометрические токены могут использоваться только для активации 
аппаратных.

\begin{figure}[bht]
\begin{center}
\includegraphics[width=9cm]{../figs/Token}
\end{center}
\caption{Токен аутентификации}
\label{Fig.TI.Token}
\end{figure}

Токены используются для построения аутентификаторов (см. 
рисунок~\ref{Fig.TI.Token}). Аутентификатор демонстрирует владение 
токеном аутентификации или его знание и, таким образом, подтверждает 
подлинность владельца.

Токены делятся на два класса: однофакторные и многофакторные.
Однофакторные токены выдают аутентификатор сразу, многофакторные~---
только после активации. Для активации нужно задействовать дополнительный 
фактор, как правило, <<что я знаю>> (PIN-код) или <<кто я>> (отпечаток пальца).

Определены следующие типы ТА:

\begin{enumerate}
\item
Статический пароль.
Секретная последовательность символов, цифр, графических элементов и пр.,
которую способен запомнить человек.
Разделяется между пользователем и СИ.
Либо выбирается пользователем, либо генерируется СИ.
Относится к фактору <<что я знаю>>. 
Аутентификатором обычно является сам пароль.

\item
Карта секретов.
Физическое или электронное устройство, которое хранит набор секретов,
разделяемых между пользователем и СИ. Относится к фактору <<чем я владею>>. 
Аутентификатором является секрет по запрошенному СИ или выбранному 
пользователем номеру.

\item
Сетевой токен.
Устройство, связанное с СИ дополнительным соединением.
Сетевой адрес устройства зафиксирован в момент регистрации.
Относится к фактору <<чем я владею>>. 
%
Аутентификатором является одноразовый секрет, например пароль, который СИ
отправляет по дополнительному соединению и предлагает ввести по основному или,
наоборот, отправляет по основному соединению и предлагает ввести по
дополнительному.
%
Как правило, дополнительное соединение~--- канал GSM, сетевой адрес~--- номер
сотового телефона пользователя.

\item
Однофакторный OTP-токен.
Аппаратное устройство или программа, которые генерируют одноразовые пароли. Для
генерации используется секретный ключ, размещенный в памяти устройства или в
файле, сопровождающем программу. Относится к фактору <<чем я владею>>. Активация
устройства не требуется.
%
% Например, токеном может быть приложение Google Authenticator, установленное 
% на смартфоне.

\item
Однофакторный аппаратный КТ.
Аппаратный криптографический токен, который не требуется активировать.
Относится к фактору <<чем я владею>>. 

\item
Программный КТ.
Программный КТ, включающий специальный файл, в котором хранится защищенный 
секрет аутентификации. Ключ защиты строится по паролю.
%
Относится к фактору <<чем я владею>>. 
Пароль является дополнительным фактором <<что я знаю>>.

\item
Многофакторный OTP-токен.
Аппаратное устройство или программа, которые генерируют одноразовые пароли. 
Для генерации используется секретный ключ, размещенный в памяти устройства,
или в файле, сопровождающем программу. 
%
Относится к фактору <<чем я владею>>. 
%
Для активации устройства нужен дополнительный фактор <<что я знаю>> или 
<<кто я>>. 
%
Ключ хранится в файле в защищенном виде. Ключ защиты файла строится по паролю.
Пароль является дополнительным фактором <<что я знаю>>.

\item
Многофакторный аппаратный КТ.
Аппаратный криптографический токен.
Относится к фактору <<чем я владею>>. 
Для активации токена нужен дополнительный фактор <<что я знаю>> 
и (или) <<кто я>>.  
\end{enumerate}

% Комбинации токенов

Пользователь может использовать несколько независимых ТА, повышая при этом 
надежность аутентификации. 

\subsection{Требования}\label{TI.Reqs}

% физический токен

\req{ВТ}{1--3}
Физический токен (карта кодов, сетевой токен, аппаратный OTP-токен или КТ)
должен сопровождаться инструкциями владельцу. Инструкции должны содержать 
сведения об обращении с токеном и о действиях в случае его пропажи или 
кражи. 

% статический пароль

\req{ВТ}{1}
Статический пароль должен содержать не менее 14 бит энтропии.
Текстовый пароль, выбираемый пользователем,
должен состоять из не менее чем 6 символов в алфавите
из 90 и более символов. Секретный PIN-код должен состоять из не менее
чем 4 цифр и выбираться СИ случайным образом. 
%
% 90 = numbers, alphabets and these 28 characters: 
% !"#$%&()*+,-./:;<=>?@[]^_{}~. 

% todo: псевдослучайным?

\req{ВТ}{2, 3}\label{R.TI.SP2}
Статический пароль должен содержать не менее 20 бит энтропии.
Текстовый пароль, выбираемый пользователем,
должен состоять из не менее чем 8 символов в алфавите
из 90 и более символов. Секретный PIN-код должен состоять из не менее
чем 6 цифр и выбираться СИ случайным образом. 

\req{ВТ}{3}
При регистрации статического пароля, выбранного пользователем, СИ должна 
проверить его отсутствие в списке паролей, которые часто используются,
ожидаемы или скомпрометированы.

% NIST SP 800-63: When processing requests to establish and change memorized 
% secrets, verifiers SHALL compare the prospective secrets against a list that 
% contains values known to be commonly-used, expected, or compromised. 

\begin{note*}
Список запрещенных паролей может включать словари распространенных паролей,
базы данных скомпрометированных паролей, тривиальные пароли из повторяющихся
или соседних символов, а также пароли, построенные по имени пользователя или 
интернет-сервера.
\end{note*}

\req{ВТ}{1--3}
КП не должна хранить подсказки, которые помогают пользователю вспомнить забытый 
статический пароль.

% NIST SP 800-63: Memorized secret verifiers SHALL NOT permit the subscriber to 
% store a “hint” that is accessible to an unauthenticated claimant. 

\begin{note*}
При этом КП может отображать подсказки, переданные от СИ или терминала 
по защищенному соединению.
\end{note*}

\req{ВТ}{1--3}
При вводе пароля его символы (по отдельности или все вместе)
могут отображаться только на короткое время,
нужное пользователю для проверки корректности ввода.

% карта секретов

\req{ВТ}{1--3}
Карта секретов должна генерироваться СИ по секретному ключу, который содержит 
не менее 128 бит энтропии.
%
Аутентификатор карты должен содержать не менее 20 бит энтропии. 
%
Аутентификатор не должен использоваться дважды. 

% A given secret from an authenticator SHALL be used successfully only once.

% сетевой токен

\req{ВТ}{1--3}
Сетевой токен должен иметь уникальный сетевой адрес,
сообщения на который может принимать только сам токен.

\begin{note*}
В частности, сетевой токен не может использовать адрес электронной почты,  
поскольку сообщения на этот адрес могут принимать почтовые клиенты на разных 
устройствах. 
%
Мессенджер, который допускает установку на нескольких устройствах, также не 
может использоваться для приема сообщений.
\end{note*}

\req{ВТ}{1--3}
Сетевой токен должен взаимодействовать с СИ по дополнительному соединению, 
отличному от основного соединения КП~-- СИ. 
Аутентификатор, который СИ передает сетевому токену по 
дополнительному соединению, должен быть зашифрован. Перед отправкой 
аутентификатора сетевой токен должен быть аутентифицирован перед СИ.
%
Для шифрования и аутентификации должны использоваться либо СКЗИ, 
удовлетворяющие требованиям СТБ 34.101.27, либо стандартные криптографические
механизмы сетей сотовой связи.

\begin{note*}
В настоящем стандарте не рассматривается сценарий, когда токен получает
один и тот же аутентификатор по основному и дополнительному соединениям,
и пользователь подтверждает совпадение аутентификаторов по дополнительному 
(защищенному) соединению.
\end{note*}

\req{ВТ}{1--3}
Аутентификатор сетевого токена должен содержать не менее 20 бит энтропии. 
%
Аутентификатор не должен действовать более 5~мин. 
%
СИ должна принимать аутентификатор лишь однажды в течение периода действия.

% NIST SP800-63-3B: In all cases, the authentication SHALL be considered 
% invalid if not completed within 10 minutes.

\req{ВТ}{1--3}
Владелец сетевого токена должен быть проинструктирован о необходимости 
настройки токена так, чтобы в неактивном (заблокированном) состоянии токен не 
отображал аутентификаторы.

\begin{note*}
Если в качестве токена выступает смартфон, то его настройка может состоять в 
подавлении отображения сообщений с аутентификаторами на экране блокировки. 
При этом уведомления о приеме сообщений можно не блокировать.
\end{note*}

% OTP-токен

\req{ВТ}{1--3}
Для генерации одноразовых паролей в OTP-токенах должны использоваться 
механизмы HOTP или TOTP, определенные в СТБ~34.101.47. 
%
В механизмах должен использоваться секретный ключ, который содержит не менее
128 бит энтропии.
%
При использовании механизма TOTP пароль должен меняться не реже одного раза в 
2~мин.
%
Пароль должен состоять из не менее чем 6 десятичных символов.

% todo: передача ключа?

\req{ВТ}{1--3}
OTP-токен должен удовлетворять требованиям СТБ~34.101.27:
программный токен~--- требованиям уровня 1 или 2,
аппаратный токен~--- требованиям уровня 3 или 4.

% todo: OTP authenticators — particularly software-based OTP generators — 
% SHOULD discourage and SHALL NOT facilitate the cloning of the secret key 
% onto multiple devices.

% КТ

\req{ВТ}{1--3}
КТ должен удовлетворять требованиям СТБ~34.101.27:
программный токен~--- требованиям уровня 1 или 2,
аппаратный токен~--- требованиям уровня 3 или 4.
%
Секрет аутентификации должен содержать не менее 128 бит энтропии,
аутентификатор~--- не менее 64 бит.

% защита от непреднамеренных действий

\req{ВТ}{1--3}
Однофакторные аппаратные OTP-токен и КТ должны выдавать аутентификаторы 
только после физического воздействия (например, нажатия кнопки).

% активация

\req{ВТ}{1--3}
Программный КТ должен активироваться статическим текстовым  
паролем, удовлетворяющим требованию~\ref{R.TI.SP2}.
По этому паролю должен строиться ключ защиты секрета аутентификации.
Должен использоваться алгоритм PBKDF2, определенный в СТБ~34.101.45,
или схожий криптографический механизм.
%
В PBKDF2 число итераций должно быть не меньше 10\,000,
битовая длина синхропосылки~--- не меньше 64.

\req{ВТ}{1--3}
Многофакторные аппаратные OTP-токен и КТ должны активироваться 
либо статическим паролем в соответствии с требованиями~СТБ~34.101.27 (пакет ИА),
либо биометрическим токеном.
%
Биометрическая аутентификация должна удовлетворять следующим условиям:
\begin{itemize}
\item
доля ошибочных решений о соответствии биометрического образца биометрическому эталону 
не превышает 1/1\,000;
% 
% The biometric system SHALL operate with an FMR [ISO/IEC 2382-37] of 1 in 1000 
% or better. This FMR SHALL be achieved under conditions of a conformant attack 
% (i.e., zero-effort impostor attempt) as defined in ISO/IEC 30107-1.
%
%\item
%предусмотрена защита от атак на биометрическое предъявление (Presentation 
%Attack Detection); 
%
\item
допускается не более 5 ошибок аутентификации подряд. При достижении порога числа
неверных попыток должна выполняться либо задержка на 30~с, либо
переход на другой фактор аутентификации. Задержка должна увеличиваться в 2 раза
с каждой новой ошибкой аутентификации.
\end{itemize}

% комбинации

\req{ВТ}{1}
Должен использоваться токен одного из допустимых типов:
статический пароль, карта секретов, сетевой токен,
OTP-токен (однофакторный или многофакторный), 
КТ (однофакторный или многофакторный, программный или аппаратный).

\req{ВТ}{2}\label{R.TI.AAL2}
Должен использоваться многофакторный OTP-токен, программный КТ, многофакторный  
аппаратный КТ или комбинация статического пароля с одним из следующих токенов:
карта секретов, сетевой токен, однофакторный OTP-токен, 
однофакторный аппаратный КТ. 
%
При повторной аутентификации в рамках аутентифицированного сеанса с СИ 
достаточно дополнительно к БС использовать статический пароль.

\begin{note*}
БС выступает в роли токена фактора <<чем я владею>>.
\end{note*}

\req{ВТ}{3}
Должен использоваться многофакторный аппаратный КТ или одна из следующих 
комбинаций токенов: 
\begin{itemize}
\item
статический пароль и однофакторный аппаратный КТ;
\item
многофакторный OTP-токен и однофакторный аппаратный КТ;
\item
однофакторный аппаратный КТ и программный~КТ;
\item
аппаратный OTP-токен (однофакторный или многофакторный) и программный~КТ.
\end{itemize}


