\begin{appendix}{Д}{обязательное}{\addendum{Объект JWT}}\label{JWT}

\hiddensection{Общие положения}

БА и защищенные ответы с узла UserInfo представляют собой объекты 
JWT (JSON Web Token, \cite{RFC7519}).
%
Используются объекты трех типов: 
\begin{itemize}
\item[1)] 
объект JWS (JSON Web Signature, \cite{RFC7515})~--- подписанные утверждения;
\item[2)] 
объект JWE (JSON Web Encryption, \cite{RFC7516})~--- конвертованные утверждения;
\item[3)] 
составной объект JWE~--- конвертованный объект JWS, т.~е. конвертованные 
подписанные утверждения.
\end{itemize}

БА представляет собой либо объект JWS, либо составной объект JWЕ.
Ответы с узла UserInfo могут быть объектами любого типа. 

Объект JWT состоит из нескольких кодовых слов, разделенных символом
\lstinline{"."} (точка). Кодовые слова описывают компоненты объекта. 
%
Компоненты представляют собой либо двоичные данные (по умолчанию, если не 
оговорено противное), либо объекты JSON.
%
Компоненты кодируются по правилам base64url~\cite{RFC4648}. 
Компоненты, которые являются объектами JSON, предварительно кодируются по 
правилам UTF8~\cite{UTF8}.

Число компонентов зависит от типа объекта. Первым компонентом всегда является 
заголовок. Он является объектом JSON.

\hiddensection{Объект JWS}\label{JWT.JWS}

Объект JWS состоит из трех компонентов:
\begin{enumerate}
\item[1)]
заголовок;
\item[2)]
подписываемые данные;
\item[3)]
подпись.
\end{enumerate}

Заголовок описывает алгоритмы выработки и проверки ЭЦП. Предусмотрены следующие 
варианты (см. определения алгоритмов и параметров в СТБ~34.101.45 и 
СТБ~34.101.77): 
\begin{itemize}
\item
\lstinline{"BIGNS128"}~--- алгоритмы \lstinline{bign-with-hbelt} с параметрами  
\lstinline{bign-curve256v1};
\item
\lstinline{"BIGNS192"}~--- алгоритмы \lstinline{bign-with-bash384} с 
параметрами \lstinline{bign-curve384v1};
\item
\lstinline{"BIGNS256"}~--- алгоритмы \lstinline{bign-with-bash512} с 
параметрами \lstinline{bign-curve512v1}.
\end{itemize}

Выбранный вариант должен быть указан в параметре \lstinline{alg} заголовка.

\begin{example*}
Заголовок объекта JWS: \lstinline|{"alg":"BIGNS128"}|.
\end{example*}

Подписываемые данные представляют собой объект JSON с утверждениями. 

Подпись вычисляется от слова, составленного из кодового представления 
заголовка, затем точки, затем кодового представления подписываемых данных. 

\hiddensection{Объект JWE}

Объект JWE состоит из следующих компонентов:
\begin{enumerate}
\item[1)] 
заголовок;
\item[2)] 
защищенный ключ (токен ключа);
\item[3)] 
синхропосылка;
\item[4)]
зашифрованные данные;
\item[5)]
имитовставка.
\end{enumerate}

Заголовок описывает алгоритмы защиты ключей и данных.

Предусмотрены следующие варианты алгоритмов защиты данных
(см. определения алгоритмов в СТБ 34.101.31):
\begin{itemize}
\item
\lstinline{"BELT-DWP"}~--- алгоритмы аутентифицированного шифрования 
\lstinline{belt-dwp}; 
\item
\lstinline{"BELT-CHE"}~--- алгоритмы аутентифицированного шифрования 
\lstinline{belt-che}.
\end{itemize}

Выбранный вариант должен быть указан в параметре \lstinline{enc} заголовка.

Предусмотрены следующие варианты алгоритмов защиты ключей
(см. определения алгоритмов и параметров в СТБ 34.101.45):
\begin{itemize}
\item
\lstinline{"BIGNT128"}~--- алгоритмы транспорта ключа 
\lstinline{bign-keytransport} с параметрами \lstinline{bign-curve256v1};
\item
\lstinline{"BIGNT192"}~--- алгоритмы транспорта ключа 
\lstinline{bign-keytransport} с параметрами \lstinline{bign-curve384v1};
\item
\lstinline{"BIGNT256"}~--- алгоритмы транспорта ключа 
\lstinline{bign-keytransport} с параметрами \lstinline{bign-curve512v1}. 
\end{itemize}

Выбранный вариант должен быть указан в параметре \lstinline{alg} заголовка. 

В заголовок составного объекта JWЕ (конвертованные подписанные данные) 
дополнительно должен быть включен параметр \lstinline{cty} со значением 
\lstinline{"JWT"}. 
%
Включение этого параметра в обычный объект JWE не рекомендуется.

\begin{example*}
Заголовки объекта JWE: 
\lstinline|{"alg":"BIGNT128","enc":"BELT-DWP"}|,
\lstinline|{"alg":"BIGNT256","enc":"BELT-CHE","cty":"JWT"}|.
\end{example*}

Ключ защиты данных должен вырабатываться случайным или псевдослучайным образом 
в соответствии с требованиями СТБ~34.101.31.
%
Защищенный ключ указывается во второй компоненте объекта JWE.

Алгоритм аутентифицированного шифрования, который используется для установки 
защиты данных, принимает на вход критические данные, ассоциированные открытые 
данные, ключ защиты данных и синхропосылку. Алгоритм возвращает зашифрованные 
данные и имитовставку.
%
Синхропосылка указывается в третьей компоненте объекта JWE, зашифрованные 
данные~--- в четвертой, имитоставка~--- в пятой.

Cинхропосылка выбирается произвольным образом.
%
Нулевая синхропосылка считается синхропосылкой по умолчанию, она может 
кодироваться пустой строкой.

Критическими данными является либо кодовое представление объекта JSON с 
утверждениями, либо вложенный объект JWS. 
%
В качестве ассоциированных открытых данных должно выступать кодовое 
представление заголовка объекта JWE.

\end{appendix}
