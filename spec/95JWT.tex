\begin{appendix}{Д}{обязательное}{Объект JWT}\label{JWT}

\hiddensection{Общие сведения}

% JSON Web Token (JWT) is a compact, URL-safe means of representing
%    claims to be transferred between two parties.  The claims in a JWT
%    are encoded as a JSON object that is used as the payload of a JSON
%    Web Signature (JWS) structure or as the plaintext of a JSON Web
%    Encryption (JWE) structure, enabling the claims to be digitally
%    signed or integrity protected with a Message Authentication Code
%    (MAC) and/or encrypted. JWTs are always
%    represented using the JWS Compact Serialization or the JWE Compact
%    Serialization.

%    A JWT is represented as a sequence of URL-safe parts separated by
%    period ('.') characters.  Each part contains a base64url-encoded
%    value.  The number of parts in the JWT is dependent upon the
%    representation of the resulting JWS using the JWS Compact
%    Serialization or JWE using the JWE Compact Serialization.

БА и защищенные ответы с узла UserInfo представляют собой объекты 
JWT (JSON Web Token) \cite{RFC7519}. 
%
Имеется 3 типа таких объектов: 
\begin{itemize}
\item[1)] 
объект JWS (JSON Web Signature) \cite{RFC7515},
который описывает подписанные данные;
\item[2)] 
объект JWE (JSON Web Encryption) \cite{RFC7516},
который описывает конвертованные данные;
\item[3)] 
вложенный объект JWT: содержит объект JWE
и для этого объекта конвертуемыми данными является объект JWS.
\end{itemize}

При защите БА используется либо объект JWS, либо вложенный объект JWT.
При защите ответов с узла UserInfo может использоваться объект любого типа. 

% Утверждения в JWT представляются JSON объектами, которые используются либо как 
% открытый текст для JSON Web Signature (JWS), либо как открытый текст для JSON Web Encryption (JWE).
Объекты JWS/JWE, в свою очередь, могут представляться двумя способами: 
компактной JWS/JWE-сериализацией и JWS/JWE-JSON-сериализацией.

Объект JWT всегда использует только компактную JWS-сериализацию (JWS 
Compact Serialization) или компактную JWE-сериализацию (JWE Compact 
Serialization). 
%
В обоих вариантах сериализации объект JWT представляется как последовательность 
закодированных по правилам base64url~\cite{RFC4648} элементов, разделенных 
символом \lstinline{"."} (точка). 
Количество элементов последовательности зависит от варианта сериализации.  

\hiddensection{Объект JWS}\label{JWT.JWS}
%  JWS represents digitally signed or MACed content using JSON data
%    structures and base64url encoding.  These JSON data structures MAY
%    contain whitespace and/or line breaks before or after any JSON values
%    or structural characters, in accordance with Section 2 of RFC 7159
%    [RFC7159].  A JWS represents these logical values (each of which is
%    defined in Section 2):
% 
%    o  JOSE Header
%    o  JWS Payload
%    o  JWS Signature

Объект JWS состоит из следующих логических компонентов:
\begin{itemize}
\item
JOSE-заголовок (JOSE Header), который задает алгоритм ЭЦП и его параметры;
\item
исходные данные (JWS Payload), которые являются объектом JSON;
\item
подпись (JWS Signature), вычисленная от JOSE-заголовка и исходных данных.
\end{itemize}

При использовании компактной JWS-сериализации 
объект JWS условно описывается следующим образом:
\begin{lstlisting}
BASE64URL(UTF8(JOSE Header)) || '.' ||
BASE64URL(JWS Payload) || '.' ||
BASE64URL(JWS Signature)
\end{lstlisting}

В зависимости от требуемого уровня стойкости для подписи данных должен использоваться 
алгоритм \lstinline{bign-with-hbelt} с параметрами \lstinline{bign-curve256v1}
или алгоритм \lstinline{bign-with-bash384} с параметрами \lstinline{bign-curve384v1} 
или алгоритм \lstinline{bign-with-bash512} с параметрами \lstinline{bign-curve512v1}. 
Алгоритм и параметры определены в СТБ 34.101.45, СТБ 34.101.77.
%
Связка \lstinline{bign-with-hbelt} с \lstinline{bign-curve256v1} обозначается 
через \lstinline{"BIGNS256"},  
связка \lstinline{bign-with-bash384} с \lstinline{bign-curve384v1}~--- 
через \lstinline{"BIGNS384"},
связка \lstinline{bign-with-bash512} c \lstinline{bign-curve512v1}~--- 
через \lstinline{"BIGNS512"}.

Примером JOSE-заголовка объекта JWS является:
\begin{lstlisting}
{"alg":"BIGNS256"}
\end{lstlisting}

\hiddensection{Объект JWE}

Объект JWE состоит из следующих логических компонентов:
\begin{itemize}
\item[--] 
JOSE-заголовок (JOSE Header), определяющий параметры и алгоритм шифрования;
\item[--] 
защищенные ключи (JWE Encrypted Key), представляющие собой 
зашифрованные ключи шифрования данных и имитовставки 
(в некоторых случаях могут представляться пустыми строками);
\item[--] 
синхропосылка (JWE Initialization Vector), используемая при шифровании 
(в некоторых случаях может представляться пустой строкой);
\item[--]
зашифрованные данные (JWE Ciphertext);
\item[--]
имитовставка (JWE Authentication Tag).
\end{itemize}

При использовании компактной JWE-сериализации объект JWE условно 
описывается следующим образом:
\begin{lstlisting}
BASE64URL(UTF8(JOSE Header)) || '.' ||
BASE64URL(JWE Encrypted Key) || '.' ||
BASE64URL(JWE Initialization Vector) || '.' ||
BASE64URL(JWE Ciphertext) || '.' ||
BASE64URL(JWE Authentication Tag)
\end{lstlisting}

Для защиты данных должен использоваться набор алгоритмов,
обозначаемый \lstinline{"BELT-CFB-MAC"}. Этот набор определяет шифрование
с помощью алгоритма~\lstinline{belt-cfb} и имитозащиту с помощью 
алгоритма~\lstinline{belt-mac}. Алгоритмы определены в СТБ 34.101.31. 
Ключи шифрования и имитозащиты вырабатываются случайным или 
псевдослучайным образом в соответствии с положениями СТБ 34.101.31.
При шифровании используется нулевая синхропосылка.
%
Ключи шифрования и имитозащиты конкатенируются, а затем защищаются 
с помощью алгоритма \lstinline{bign-keytransport},
который выполняется с выбранными параметрами подписи.  
Алгоритм и параметры определены в СТБ 34.101.45.
%
Защита ключей производится на открытом ключе получателя. 
%
Связка \lstinline{bign-keytransport} с \lstinline{bign-curve256v1}
обозначается через \lstinline{"BIGNT256"},
с \lstinline{bign-curve384v1}~--- через \lstinline{"BIGNT384"},
c \lstinline{bign-curve512v1}~--- через \lstinline{"BIGNT512"}.

Примером JOSE-заголовка объекта JWE является:
\begin{lstlisting}
 {"alg":"BIGNT256","enc":"BELT-CFB-MAC"}
\end{lstlisting}

\hiddensection{Различия между объектами JWS и JWE}

%    There are several ways of distinguishing whether an object is a JWS
%    or JWE.  All these methods will yield the same result for all legal
%    input values; they may yield different results for malformed inputs.

Существует несколько методов определения принадлежности объекта к JWS или JWE
(все описанные ниже методы будут показывать одинаковый результат 
для всех возможных корректных данных,
однако для некорректных данных результат не определен):

%    o  If the object is using the JWS Compact Serialization or the JWE
%       Compact Serialization, the number of base64url-encoded segments
%       separated by period ('.') characters differs for JWSs and JWEs.
%       JWSs have three segments separated by two period ('.') characters.
%       JWEs have five segments separated by four period ('.') characters.
%       
%       The JOSE Header for a JWS can be distinguished from the JOSE
%       Header for a JWE by examining the "alg" (algorithm) Header
%       Parameter value.  If the value represents a digital signature or
%       MAC algorithm, or is the value "none", it is for a JWS; if it
%       represents a Key Encryption, Key Wrapping, Direct Key Agreement,
%       Key Agreement with Key Wrapping, or Direct Encryption algorithm,
%       it is for a JWE.  (Extracting the "alg" value to examine is
%       straightforward when using the JWS Compact Serialization or the
%       JWE Compact Serialization and may be more difficult when using the
%       JWS JSON Serialization or the JWE JSON Serialization.)
% 
%    o  The JOSE Header for a JWS can also be distinguished from the JOSE
%       Header for a JWE by determining whether an "enc" (encryption
%       algorithm) member exists.  If the "enc" member exists, it is a
%       JWE; otherwise, it is a JWS.

\begin{itemize}
\item[--] 
при использовании компактной сериализации, количество сегментов,
закодированных согласно base64url и разделенных точкой различно для объектов 
JWS и JWE:  
объект JWS состоит из трех фрагментов, разделенных двумя точками, в то время как 
объект JWE состоит из пяти фрагментов, разделенных четырьмя точками;
\item[--] 
определить тип объекта можно анализируя поле \lstinline{alg} в JOSE-заголовке. 
Если его значение определяет алгоритм ЭЦП, или его значение 
\lstinline{"none"}, то это объект JWS;
\item[--] 
если в JOSE-заголовке есть поле \lstinline{enc} (алгоритм шифрования), 
то это объект JWE, иначе --- объект JWS.
\end{itemize}

\hiddensection{JOSE-заголовок}

JOSE-заголовок (JOSE Header)~--- это объект JSON, параметры которого
описывают криптографические операции,
применяемые к набору JWT-утверждений (JWT Claims Set), и, необязательно, 
дополнительные свойства объекта JWT. 

Перечень стандартных параметров заголовка определен в 
спецификациях~\cite{RFC7515,RFC7516,RFC7519}.
%
В настоящем стандарте используются только параметры \lstinline{alg} и 
\lstinline{enc}. 
%
% todo: используются vs применяются?

В зависимости от того, чем объект JWT является (объектом JWS или объектом JWE), 
применяются те или иные правила формирования JOSE-заголовка.

Если JOSE-заголовок используется для объекта JWS, то объект JWT представляется в
виде объекта JWS и утверждения подписываются, т.е. JWT-утверждения являются
исходными данными для объекта JWS (исходные данные подписываются совместно с
заголовком).

Если JOSE-заголовок используется для объекта JWE, то объект JWT представляется в
виде объекта JWE и утверждения зашифровываются, т.е. JWT-утверждения являются
исходными данными, защищаемыми на уровне объекта JWE.

Объект JWT может включаться в другую JWE или JWS структуру, создавая, таким
образом, вложенный объект JWT.

\end{appendix}
