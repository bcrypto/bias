\begin{appendix}{Е}{обязательное}{Билет аутентификации OIDC}\label{IDTOKEN}

\hiddensection{Структура билета}\label{IDTOKEN.Structure}

БА включает утверждения аутентификации, определенные в \ref{CLAIMS.Auth}.
БА может дополнительно включать утверждения о пользователе, 
определенные в~\ref{CLAIMS.User}, а также другие утверждения.
% 
Перечень утверждений, которые следует включить в БА, уточняется в 
запросе авторизации~/ аутентификации (см.~\ref{CLAIMS.ReqWith}).

Пример набора утверждений БА:
\begin{lstlisting}
{
 "iss": "https://server.example.com",
 "sub": "24400320",
 "aud": "s6BhdRkqt3",
 "nonce": "n-0S6_WzA2Mj",
 "exp": 1311281970,
 "iat": 1311280970,
 "auth_time": 1311280969,
 "acr": "30"
}
\end{lstlisting}

\addendum{
БА представляет собой объект JWT, в котором утверждения подписываются и возможно 
дополнительно конвертуются.}

% Если БА возвращается с узла Token, то параметр \lstinline{alg} в заголовке 
% билета как объекта JWT не должен принимать значение \lstinline{"none"}.
%
% Исключение может быть сделано, если ПС при регистрации согласовал использование 
% \lstinline{"none"}.
%
% В заголовке БА не следует использовать параметры 
% \lstinline{x5u}, \lstinline{x5c}, \lstinline{jku} или \lstinline{jwk}
% (см.~\cite{RFC7515,RFC7516}).

\hiddensection{Обработка билета}\label{IDTOKEN.Process}

ПС должна обработать полученный БА следующим образом.

\begin{enumerate}
\item 
Если БА конвертован, то снять с него защиту, используя ключи и алгоритмы, 
которые ПС определила для этих целей при регистрации.
%
Если во время регистрации ПС конвертование билетов было для нее
предусмотрено, но БА не конвертован, то его необходимо отклонить.

\item 
Проверить подпись БА. При проверке ПС должна использовать открытые ключи, 
предоставленные ей при регистрации или полученные другим доверенным способом. 

\item 
Если БА получен напрямую от аутентифицированной СИ (с узла Token), 
то проверка подписи БА может опускаться.

\item  
Проверить, что идентификатор СИ как издателя БА совпадает с идентификатором, 
указанным в утверждении \lstinline{iss}.

\item 
Проверить, что утверждение \lstinline{aud} включает идентификатор 
\lstinline{client_id}, назначенный ПС при регистрации.
%
БА должен быть отвергнут, если идентификатор ПС не включен в \lstinline{aud}
или если \lstinline{aud} содержит идентификатор стороны, которой ПС не 
доверяет. 

\item 
Если утверждение \lstinline{aud} содержит несколько идентификаторов, 
то следует проверить, что утверждение \lstinline{azp} включено в БА. 

\item  
Если утверждение \lstinline{azp} включено в БА, то следует проверить, 
что идентификатор \lstinline{client_id}, который был назначен ПС при 
регистрации, совпадает с идентификатором в \lstinline{azp}.

\item 
Проверить, что текущее время предшествует времени в утверждении 
\lstinline{exp}. 

\item 
Утверждение \lstinline{iat} может быть использовано для отклонения
билетов, которые были выпущены слишком давно в сравнении с текущим временем.
Допустимое превышение времени может являться специфическим 
для каждой ПС и определяется с учетом времени хранения на ПС синхропосылок 
запроса авторизации~/ аутентификации.

\item
Если утверждение \lstinline{jti} включено в БА, то проверить, что 
представленный в нем идентификатор не использовался в других доступных
билетах.

\item 
Проверить наличие в БА утверждения \lstinline{nonce}: оно должно 
присутствовать, если билет возвращен с узла Authorization или если билет 
возвращен с узла Token и запрос авторизации~/ аутентификации включал параметр 
\lstinline{nonce}.
%
При наличии утверждения \lstinline{nonce} сравнить указанную в нем 
синхропосылку с синхропосылкой в параметре \lstinline{nonce} запроса.
%
Если обнаружено несовпадение синхропосылок или если сравнение не может быть 
выполнено (например, по причине истечения срока хранения синхропосылок) или если
ПС обнаруживает, что БА с данной синхропосылкой уже обрабатывался, 
то отклонить билет.
%
ПС может использовать дополнительные методы противодействия атакам, основанным 
на повторах БА. 

\item Если запрашивалось утверждение \lstinline{acr}, то проверить,
что запрос корректно обработан (см.~\ref{CLAIMS.Acr}). 

\item Если запрашивалось утверждение \lstinline{auth_time} 
(через параметр \lstinline{claims} или \lstinline{max_age}), 
то проверить, что запрос корректно обработан.
%
Проанализировать момент аутентификации пользователя,
указанный в утверждении, и запросить повторную аутентификацию, 
если время, которое прошло с этого момента, больше допустимого.
\end{enumerate}

Если какая-либо из проверок завершается с ошибкой, то БА должен быть отклонен.

Любые утверждения, которые содержатся в БА
и не могут быть распознаны, должны игнорироваться ПС.
 
\hiddensection{Обновление билета}\label{IDTOKEN.Refresh}

При выпуске БА в результате обработки запроса на обновление должны соблюдаться 
следующие правила.

\begin{enumerate}
\item
Утверждения \lstinline{iss}, \lstinline{sub}, \lstinline{aud}
в первоначальном и обновленном БА должны совпадать.

\item
В утверждении \lstinline{iat} должен быть указан момент выпуска обновленного БА. 

\item
Если первоначальный БА содержал утверждение \lstinline{auth_time}, то оно
должно быть повторено в обновленном БА.

\item
Если первоначальный БА содержал утверждение \lstinline{azp}, то оно
должно быть повторено в обновленном БА.
%
Если утверждение \lstinline{azp} отсутствовало в первоначальном БА, то оно 
должно отсутствовать в обновленном.
\end{enumerate}

В остальном правила выпуска обновленного БА не отличаются от правил выпуска 
первоначального.

\end{appendix}

