\section{Назначение}\label{COMMON.Purpose}

Аутентификация, проверка подлинности, является одной из услуг доверия. 
%
Пользователь, прошедший аутентификацию, приобретает <<цифровой образ>> 
(digital identity). Образ представляет пользователя в информационной системе, 
является его виртуальным посредником при доступе к сервисам системы.
%
Для сервисов, к которым пользователь обращается многократно, обычно 
гарантируется неизменность образа при повторных обращениях. 
%
Образ при этом становится устойчивым, его принимают другие стороны 
и отождествляют с владельцем.

Аутентификации предшествует идентификация. Речь идет о регистрации пользователя
в системе с назначением ему уникального идентификатора. В процессе
идентификации, как правило, \addendum{подтверждается личность} регистрируемого
пользователя, сохраняются его идентификационные данные.

Аутентификация продолжается распространением утверждений аутентификации в
пределах федерации, т.е. среди сторон системы, связанных отношениями 
доверия.
%
Эти отношения позволяют организовать надежную передачу информации о подлинности 
пользователей, прошедших аутентификацию, и, таким образом, создавать новые отношения 
доверия.
%
Распространяемые утверждения могут сопровождаться разрешениями
аутентифицированной стороны на доступ к собственным ресурсам, в том числе
идентификационным данным. Другими словами, аутентификация в федерации может
продолжаться авторизацией.

%Трансляция может выполняться псевдонимно, с передачей минимального числа 
%идентификационных атрибутов, вплоть до исключительно утверждений 
%аутентификации.

Настоящий стандарт устанавливает правила построения инфраструктур 
аутентификации, составленных из трех описанных выше блоков: 
идентификация (раздел~\ref{ID}), аутентификация (раздел~\ref{AUTH}), 
федерация (раздел~\ref{FED}).
%           
Устанавливаемые правила соответствуют 
стандартам~\cite{SP800-63-3,SP800-63-3A,SP800-63-3B,SP800-63-3C,ISO29115}. 

Дополнительно в разделе~\ref{OIDC} представлена реализация элементов 
аутентификации и федерации на основе технологии OIDC~\cite{OIDC}, в свою 
очередь основанной на технологии OAuth 2.0~\cite{RFC6749}. В разделе 
унифицируются интерфейсы веб-сервисов, предназначенных для развертывания
в Интернете крупных открытых инфраструктур аутентификации.
%
Технология OIDC детализируется в приложениях~\ref{REQRESP}~--- \ref{IDTOKEN}. 
%
Приложения~\ref{REQRESP}~--- \ref{IDTOKEN} являются обязательными только при 
условии использования технологии.

Настоящий стандарт ориентирован на построение инфраструктур, которые обладают
следующими свойствами и характеристиками.

\begin{enumerate}
\item
Централизованная идентификация.
%
Пользователи, которые регистрируются для доступа к услуге аутентификации, 
идентифицируются одной или несколькими СИ. 
%
Небольшое число СИ могут обслужить большое число пользователей.

\item
Распределенное подтверждение личности.
%
Подтверждение проводят РЦ, обслуживающие локальные (например, 
территориальные или ведомственные) группы регистрируемых пользователей. 
%
Подтверждение личности как услуга может быть сделано максимально доступным для 
пользователей. 

\item
Централизованное управление идентификационными данными.
%
РЦ передает СИ результаты подтверждения личности и, в случае успеха, 
идентификационные данные проверенного пользователя.
%
Идентификационные данные централизованно хранятся на серверах СИ, что 
снижает издержки и угрозы раскрытия.

\item
Централизованная аутентификация.
%
ПС, взаимодействующие с пользователями и заинтересованные в проверке 
их подлинности, получают утверждения аутентификации от СИ, на услуги которых 
они подписаны.
%
ПС могут сосредоточиться на своих функциональных обязанностях, а не на 
непрофильной аутентификации.

\item
Масштабируемая аутентификация.
%
СИ может делегировать организацию аутентификации терминалам, выступающим в роли
агентов СИ. Использование терминалов позволяет масштабировать нагрузку при
доступе к услуге аутентификации, повышает степень проникновения услуги.

\item
Централизованная авторизация. 
%
Вместе с утверждениями аутентификации ПС получают от СИ авторизационные 
разрешения на доступ к ресурсам аутентифицированных пользователей. Ресурсы 
размещаются централизованно на выделенных СР, напрямую взаимодействующих со СИ.
\end{enumerate}

%В настоящем стандарте  не рассматриваются вопросы идентификации и 
%аутентификации для доступа к сервисам СКЗИ. Эти вопросы рассмотрены в 
%СТБ~34.101.27 (пакет ИА). 
