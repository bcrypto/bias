\section{Узел~Authorization}\label{OIDC.Authorization}

Узел Authorization обрабатывает запросы авторизации~/ аутентификации. 
%
К узлу обращается КП пользователя по направлению ПС.
%
При обработке запроса СИ проводит аутентификацию пользователя, получает 
согласие пользователя на доступ ПС к его ресурсам, возможно выпускает код 
авторизации, перенаправляет пользователя к ПС. 

Адрес узла должен быть зафиксирован в документах инфраструктуры
или адрес должен быть сообщен ПС во время регистрации в инфраструктуре.
%
Адрес не должен содержать компонент \lstinline{fragment}.

Запрос на узел и соответствующие ответы описаны в 
таблице~\ref{Table.OIDC.Authorization}.
%
Запрос и ответы могут включать параметры, дополнительные к перечисленным в 
таблице. 

Узел должен поддерживать HTTP-метод GET и может поддерживать метод POST. 
%
При использовании GET запрос должен кодироваться по схеме Query,
при использовании POST~--- по схеме Form.
%
Нераспознанные параметры запроса и параметры без значений должны 
игнорироваться.
%
Параметры не должны повторяться.

Ответ узла должен кодироваться по схеме Query или Fragment с перенаправлением 
на узел Redirection. 
%
Схема кодирования Query должна применяться только в коммуникационной схеме Code.
%
Для перенаправления в ответе должен указываться код 302 HTTP.

\begin{table}[p]
\caption{Узел Authorization: запрос и ответы}
\label{Table.OIDC.Authorization} 
\begin{tabular}{|l|c|p{9.0cm}|l|}
\hline
\multirow{2}{*}{Параметр} & Вклю- & \multirow{2}{*}{Описание} & \multirow{2}{*}{Ссылка}\\
                          & чение &&\\
\hline
\hline
\multicolumn{4}{|c|}{Запрос (GET/Query или POST/Form)}\\
\hline
\hline
%
\lstinline!scope! & $+$ & 
Запрашиваемая область действия & 
\ref{PARAMS.Scope}\\
\hline
%
\lstinline!response_type! & $+$ & 
Тип ответа & 
\ref{PARAMS.RespType}\\
\hline
%
\lstinline!client_id! & $+$ & 
Идентификатор ПС & 
\ref{PARAMS.ClientId}\\
\hline
%
\lstinline!redirect_uri! & $+$ &
Адрес перенаправления ответа & 
\ref{PARAMS.RedirectUri}\\
\hline
%
\lstinline!state! & р & 
Данные для перенаправления & 
\ref{PARAMS.State}\\
\hline
%
\lstinline!response_mode! & о & 
Механизм возвращения ответа &
\ref{PARAMS.RespMode}\\
\hline
%
\lstinline!nonce! & у & 
Волатильные данные для переноса в БА &
\ref{PARAMS.Nonce}\\
\hline
%
\lstinline!display! & о & 
Тип интерфейса пользователя &
\ref{PARAMS.Display}\\
\hline
%
\lstinline!prompt! & о & 
Перечень приглашений пользователю &
\ref{PARAMS.Prompt}\\
\hline
%
\lstinline!max_age! & о & 
Срок действия утверждений аутентификации &  
\ref{PARAMS.MaxAge}\\
\hline
%
\lstinline!ui_locales! & о &
Предпочтительные языки интерфейса пользователя & 
\ref{PARAMS.UiLocales}\\
\hline
%
\lstinline!id_token_hint! & o & 
Ранее выпущенный БА & 
\ref{PARAMS.IdTokenHint}\\
\hline
%
\lstinline!login_hint! & o &
Подсказка об идентификаторе пользователя & 
\ref{PARAMS.LoginHint}\\
\hline
%
\lstinline!acr_values! & o & 
Предпочтительные уровни гарантий аутентификации & 
\ref{PARAMS.AcrValues}\\
\hline
%
\lstinline!claims! & o & 
Перечень утверждений для включения в БА & 
\ref{PARAMS.Claims}\\
\hline
%
\lstinline!claims_locales! & o & 
Предпочтительные языки утверждений & 
\ref{PARAMS.ClaimsLocales}\\
\hline
%
\lstinline!request! & o & 
Защищенный запрос в виде объекта JWT & 
\ref{PARAMS.Request}\\
\hline
%
\lstinline!request_uri! & o & 
Ссылка на защищенный запрос & 
\ref{PARAMS.RequestUri}\\
\hline
%
\hline
\multicolumn{4}{|c|}{Успешный ответ в схеме Code (Query)}\\ 
\hline
\hline
%
\lstinline!code! & $+$ & 
Код авторизации & 
\ref{PARAMS.Code}\\
\hline
%
\lstinline!state! & у & 
Копия параметра \lstinline!state! запроса &  
\ref{PARAMS.State}\\
\hline
%
\hline
\multicolumn{4}{|c|}{Успешный ответ в схеме Implicit (Fragment)}\\
\hline
\hline
%
\lstinline!access_token! & у & 
БД & 
\ref{PARAMS.AccessToken}\\
\hline
%
\lstinline!token_type! & у & 
Тип БД &
\ref{PARAMS.TokenType}\\
\hline
%
\lstinline!expires_in! & p & 
Срок действия БД &
\ref{PARAMS.ExpiresIn}\\
\hline
%
\lstinline!id_token! & $+$ & 
БА &
\ref{PARAMS.IdToken}\\
\hline
%
\lstinline!scope! & у & 
Разрешенная область действия & 
\ref{PARAMS.Scope}\\
\hline
%
\lstinline!state! & у & 
Копия параметра \lstinline!state! запроса &  
\ref{PARAMS.State}\\
\hline
%
\hline
\multicolumn{4}{|c|}{Успешный ответ в схеме Hybrid (Fragment)}\\
\hline
\hline
%
\lstinline!code! & $+$ & 
Код авторизации & 
\ref{PARAMS.Code}\\
\hline
%
\lstinline!access_token! & у & 
БД & 
\ref{PARAMS.AccessToken}\\
\hline
%
\lstinline!token_type! & у & 
Тип БД &
\ref{PARAMS.TokenType}\\
\hline
%
\lstinline!expires_in! & р & 
Срок действия БД &
\ref{PARAMS.ExpiresIn}\\
\hline
%
\lstinline!id_token! & у & 
БА &
\ref{PARAMS.IdToken}\\
\hline
%
\lstinline!scope! & у & 
Разрешенная область действия & 
\ref{PARAMS.Scope}\\
\hline
%
\lstinline!state! & у & 
Копия параметра \lstinline!state! запроса &  
\ref{PARAMS.State}\\
\hline
%
\hline
\multicolumn{4}{|c|}{Ответ об ошибке (Query или Fragment)}\\
\hline
\hline
%
\lstinline!error! & $+$ & 
Код ошибки & 
\ref{PARAMS.Error}\\
\hline
%
\lstinline!error_description! & о & 
Описание ошибки &
\ref{PARAMS.ErrorDescr}\\
\hline
%
\lstinline!error_uri! & о & 
Адрес веб-страницы с информацией об ошибке &
\ref{PARAMS.ErrorUri}\\
\hline
%
\lstinline!state! & у & 
Копия параметра \lstinline!state! запроса &  
\ref{PARAMS.State}\\
\hline
\end{tabular}
\end{table}

В параметре \lstinline{response_type} запроса на узел
задается коммуникационная схема, которую требуется использовать.
%
Возможные значения параметра и соответствующие им коммуникационные схемы 
представлены в таблице~\ref{Table.OIDC.RespType}. 
%
В лексемах параметра перечисляются ожидаемые в ответе с узла 
Authorization билеты: \lstinline{"code"} указывает на код 
авторизации, \lstinline{"id_token"}~--- на БА, \lstinline{"token"}~--- на БД.

\begin{table}[hbt]
\caption{Выбор коммуникационной схемы}\label{Table.OIDC.RespType}
\begin{tabular}{|l|l|}
\hline
Значение параметра \lstinline!response_type! & Схема\\
\hline
\hline
\lstinline!"code"! & Code \\
\lstinline!"id_token"! & Implicit \\
\lstinline!"id_token token"! & Implicit \\
\lstinline!"code id_token"! & Hybrid \\
\lstinline!"code token"! & Hybrid \\
\lstinline!"code id_token token"! & Hybrid \\
\hline
\end{tabular}
\end{table}

