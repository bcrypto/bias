\section{Стороны}\label{COMMON.Entities}

Определены следующие роли сторон инфраструктуры аутентификации.

\begin{enumerate}
\item
СИ. Проводит идентификацию пользователя, передает идентификационные данные 
пользователя и другие его ресурсы СР. 
%
По запросу ПС проводит аутентификацию пользователя.
%
По согласованию с пользователем авторизует доступ ПС к его ресурсам.
%
Управляет сетью терминалов, которые участвуют в аутентификации.

\item
Терминал. По поручению СИ организует аутентификацию пользователей
или самостоятельно проводит аутентификацию и сообщает результат СИ. 

\item
РЦ. Является посредником при взаимодействии между пользователями и СИ во время 
регистрации: проводит сбор идентификационных данных пользователей и подтверждение
их личности, передает идентификационные данные СИ. 

\item
СР. Хранит идентификационные данные пользователей, управляет другими 
их ресурсами. Может входить в состав СИ.

\item
ЦФ. Управляет отношениями доверия в федерации, связанной с инфраструктурой
аутентификации.

\item
Пользователь. Регистрируется с помощью РЦ, передает идентификационные данные и 
другие ресурсы для размещения на СР. По запросу ПС проходит аутентификацию 
перед ПС. При аутентификации авторизует ПС на доступ к своим ресурсам.

% Ресурсы пользователя включают его идентификационные данные: 
% полное имя, адрес, дату рождения, номер телефона и др.
%
% Кроме этого, в состав ресурсов могут входить настройки личного кабинета, 
% история транзакций, личные файлы-документы.

\item
ПС. Регистрируется в федерации для доступа к услуге аутентификации.
Организует аутентификацию пользователя перед СИ. После авторизационного
разрешения пользователя получает доступ к его ресурсам, размещенным на СР.
\end{enumerate}

В настоящем стандарте ЦФ представляет ИОК, в которой стороны инфраструктуры 
аутентификации (возможно за исключением пользователей) получают сертификаты 
открытых ключей. 
%
Проверяя сертификат открытого ключа стороны~$A$ и убеждаясь в знании ею 
соответствующего личного ключа, сторона~$B$ убеждается в подлинности~$A$,
возникают отношения доверия.
%
Кроме этого, сертификаты могут использоваться для создания защищенных 
соединений между сторонами, например, соединений протокола TLS, 
установленного в СТБ 34.101.65. 
%
Наконец сертификаты могут использоваться для проверки подписи билетов, для их 
конвертования, в других целях.  

Инфраструктуры аутентификации настоящего стандарта совместимы с ИОК, 
установленными в СТБ 34.101.78.

Функциональные возможности ЦФ могут выходить за рамки стандартных сервисов ИОК. 
Например, ЦФ может выпускать в обращение персональные аппаратные КТ 
пользователей. При выпуске на КТ записываются идентификационные данные, личный 
ключ и сертификат владельца. КТ могут использоваться при аутентификации.

