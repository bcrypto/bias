\section{Узел UserInfo}\label{OIDC.UserInfo}

Узел UserInfo обрабатывает запросы на предоставление сведений 
об аутентифицированном пользователе. 
% 
К узлу обращается ПС, предъявляя ранее полученный БД.
%
СИ проверяет билет и, в случае успеха, возвращает утверждения о пользователе.

% todo: В адресе узла должна использоваться схема https.

Запрос на узел и соответствующие ответы описаны в 
таблице~\ref{Table.OIDC.UserInfo}.
%
Запрос и ответы могут включать параметры, дополнительные к перечисленным в 
таблице. 

\begin{table}[H]
\caption{Узел UserInfo: запрос и ответы}\label{Table.OIDC.UserInfo}  
\begin{tabular}{|l|c|p{8.9cm}|l|}
\hline
\multirow{2}{*}{Параметр} & Вклю- & \multirow{2}{*}{Описание} & \multirow{2}{*}{Ссылка}\\
                          & чение &&\\
\hline
\hline
\multicolumn{4}{|c|}{Запрос (GET/Bearer или POST/Bearer)}\\
\hline
\hline
%
\lstinline!access_token! & + & 
БД & 
\ref{PARAMS.AccessToken}\\
\hline
%
\hline
\multicolumn{4}{|c|}{Успешный ответ (JSON или JWT)}\\
\hline
\hline
%
\lstinline!sub! & + &
Идентификатор пользователя & 
\ref{CLAIMS.Sub}\\
\hline
%
\lstinline!iss! & ру & 
Идентификатор СИ & 
\ref{CLAIMS.Iss}\\
\hline
%
\lstinline!aud! & ру & 
Идентификатор ПС & 
\ref{CLAIMS.Aud}\\
\hline
%
\hline
\multicolumn{4}{|c|}{Ответ об ошибке (WWW-Authenticate)}\\
\hline
\hline
%
\lstinline!error! & ру & 
Код ошибки & 
\ref{PARAMS.Error}\\
\hline
%
\lstinline!error_description! & о & 
Описание ошибки &
\ref{PARAMS.ErrorDescr}\\
\hline
%
\lstinline!error_uri! & о & 
Адрес веб-страницы с информацией об ошибке &
\ref{PARAMS.ErrorUri}\\
\hline
%
\lstinline!scope! & o & 
Требуемая область действия & 
\ref{PARAMS.Scope}\\
\hline
\end{tabular}
\end{table}


Узел должен поддерживать HTTP-методы GET и POST. Для обращения к узлу 
рекомендуется использовать метод GET. 
%
Отправляемый БД должен кодироваться по схеме Bearer.

Успешный ответ с утверждениями о пользователе должен кодироваться по схеме JSON 
и JWT. 
%
Схема JWT применяется тогда, когда утверждения подписываются и (или) 
конвертуются. 
%
Утверждения могут только подписываться, могут только конвертоваться и могут 
сначала подписываться, а затем конвертоваться.

Ответ об ошибке должен кодироваться по схеме WWW-Authenticate.
