\chapter{Нормативные ссылки}\label{Refs}

%В настоящем стандарте использованы ссылки на следующие технические 
%нормативные правовые акты в области технического нормирования 
%и стандартизации (далее~--- ТНПА): 

СТБ~34.101.19-2012 Информационные технологии и безопасность. Форматы
сертификатов и списков отозванных сертификатов инфраструктуры открытых ключей

СТБ~34.101.23-2012 Информационные технологии и безопасность. Синтаксис
криптографических сообщений

СТБ~34.101.27-2022 Информационные технологии и безопасность. Средства
криптографической защиты информации. Требования безопасности

СТБ~34.101.31-2020 Информационные технологии и безопасность. Алгоритмы 
шифрования и контроля целостности  

СТБ~34.101.45-2013 Информационные технологии и безопасность. Криптографические
алгоритмы электронной цифровой подписи и транспорта ключа на основе
эллиптических кривых

СТБ~34.101.47-2017 Информационные технологии и безопасность. Криптографические
алгоритмы генерации псевдослучайных чисел

СТБ~34.101.65-2013 Информационные технологии и безопасность. Протокол защиты
транспортного уровня (TLS)

%СТБ~34.101.66-2013 Информационные технологии и безопасность. 
%Протоколы формирования общего ключа на основе эллиптических кривых

СТБ~34.101.77-2020 Информационные технологии и безопасность. Криптографические
алгоритмы на основе sponge-функции

СТБ~34.101.78-2019 Информационные технологии и безопасность. Профиль
инфраструктуры открытых ключей

% ГОСТ 27463-87 Система обработки информации. 7-битные кодированные наборы 

\begin{note*}
При пользовании настоящим стандартом целесообразно проверить действие
ссылочных документов на официальном сайте Национального фонда
технических нормативных правовых актов в глобальной компьютерной сети Интернет.

Если ссылочные документы заменены (изменены), то при пользовании настоящим
стандартом следует руководствоваться действующими взамен документами. Если
ссылочные документы отменены без замены, то положение, в котором дана ссылка на
них, применяется в части, не затрагивающей эту ссылку.
\end{note*}

