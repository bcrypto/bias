\thispagestyle{empty}

\noindent
{\bf ГОСУДАРСТВЕННЫЙ СТАНДАРТ} \hfill {\bf\draftlogo}\\
\noindent
{\bf РЕСПУБЛИКИ~БЕЛАРУСЬ}\\[-9pt]
\hrule height 1pt
\vskip0.4mm
\hrule height 2pt

\vskip2cm
\noindent
{\bf\Large Информационные технологии и безопасность}\\[10pt]
{\bf\large ИНФРАСТРУКТУРЫ АУТЕНТИФИКАЦИИ}

\vskip2cm
\noindent
{\bf\Large Iнфармацыйныя тэхналогii i бяспека}\\[10pt]
{\bf\large ІНФРАСТРУКТУРЫ АЎТЭНТЫФІКАЦЫІ}

\vskip12cm
\hrule height 1pt
\vskip0.4mm
\hrule height 2pt
\noindent
\begin{tabular}{p{5cm}cp{4cm}}
\vtop{\null\hbox{{\includegraphics[width=2.6cm]{../figs/stb}}}} & \hspace{6cm} & 
\mbox{}\newline\mbox{}\newline\newline Госстандарт\newline Минск\\
\end{tabular}

\pagebreak


\hrule
\vskip2mm

УДК~004.4.056.55(083.74)(476)\hfill
МКС~35.240.40\hfill
КП~05

\vskip0.5mm

{\bf Ключевые слова}: информационные технологии, безопасность,
аутентификация, идентификация, федерация доверия, требования безопасности,
технология OIDC

\vskip0.5mm

\hrule 

\rule{0pt}{5mm}
	 
\centerline{\bf Предисловие} 

Цели, основные принципы, положения по государственному регулированию и 
управлению в области технического нормирования и стандартизации 
установлены Законом Республики Беларусь <<О техническом нормировании и 
стандартизации>>.  

\vskip0.2cm

1~РАЗРАБОТАН учреждением Белорусского государственного университета
<<Науч\-но-исследовательский институт прикладных проблем математики и информатики>>

2~УТВЕРЖДЕН и ВВЕДЕН В ДЕЙСТВИЕ постановлением Госстандарта Республики 
Беларусь от $\phantom{\text{25 ноября 2011 г.}}$ \No~$\phantom{\text{83}}$ 

3~ВВЕДЕН ВПЕРВЫЕ

\vfill

\hrule
\vskip1mm
Издан на русском языке

\pagebreak
