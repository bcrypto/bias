\section{Пакеты}\label{COMMON.Packets}

Соответствие тому или иному уровню гарантий реализуется через выполнение 
требований безопасности, определенных в настоящем стандарте.

Требования безопасности схожего назначения группируются в пакеты. 
Пакет~--- это также процедуры, процессы и элементы, которых касаются требования.
%
Перечень пакетов для каждого из блоков представлен в таблице~\ref{Table.COMMON.Packages}.

\begin{table}[hbt]
\caption{Пакеты}\label{Table.COMMON.Packages}
\begin{tabular}{|l|l|c|c|}
\hline
Блок & \multicolumn{1}{c|}{Пакет} & Код & Ссылка\\
\hline
\hline
\multirow{2}{*}{Идентификация} 
  & Регистрация пользователей & РП & \ref{UR}\\
  & Подтверждение личности    & ПЛ & \ref{IP}\\
\hline
\hline
\multirow{3}{*}{Аутентификация} 
  & Выпуск токенов            & ВТ & \ref{TI}\\
  & Управление аттестатами    & УА & \ref{CM}\\
  & Протоколы аутентификации  & ПА & \ref{AP}\\
\hline
\hline
\multirow{3}{*}{Федерация} 
  & Управление федерацией     & УФ & \ref{FM}\\
  & Управление билетами       & УБ & \ref{TM}\\
  & Управление сеансами       & УС & \ref{SM}\\
\hline
\end{tabular}
\end{table}

В преамбуле пакета определяется его назначение, дается обзор понятий, 
необходимых для формулировки требований пакета и правильной их интерпретации. 

Требования внутри пакета нумеруются последовательно, начиная с единицы. Номер
требования через точку присоединяется к коду пакета, в результате получается полное
имя требования: РП.1, УБ.2 и т.д.

Для каждого требования в круглых скобках перечисляются уровни, на которых 
требование выдвигается, например: (3), (1–3), (2, 3). 

%В тексте имеются ссылки на требования. Для ссылки на сущности, введенные в 
%требовании T, используется форма [T] (прямая ссылка). Для детализации T с 
%учетом условий и пояснений в месте ссылки используется форма {T} (обратная 
%ссылка). 

%Формулировки требований безопасности могут включать указания на необходимость 
%определения списка методов, списка компонентов, списка механизмов и др. Если не 
%оговорено противное, то результатом определения может быть пустой список. 

\if0
Аутентификации пользователя перед СИ предшествуют, 
ее реализуют и сопровождают следующие процедуры (см.~рис.~\ref{Fig.AUTH.Auth}): 
\begin{enumerate}
\item
Регистрация пользователей (РП).

\item
Управление токенами (УТ).

\item
Протоколы аутентификации (ПА).

\item
Управление билетами (УБ).

\item
Управление аттестатами (УА).
\end{enumerate}

\begin{figure}[bht]
\begin{center}
\includegraphics[width=15cm]{../figs/Auth}
\end{center}
\caption{Аутентификация}
\label{Fig.AUTH.Auth}
\end{figure}

Вторая процедура определяет перечень токенов, которые получает пользователь,
содержимое токенов, правила хранения и применения токенов.

Третья процедура определяет способ аутентификации пользователя 
при непосредственном взаимодействии СИ и~КП.
%
В процессе аутентификации стороны обмениваются 
определенными сообщениями, выполняют определенные расчеты и проверки.
%
Все это и является содержанием протокола аутентификации.

Четвертая процедура определяет правила выпуска билетов после 
аутентификации или предъявления других билетов, 
правила хранения и использования билетов. 

Наконец, пятая процедура касается управления аттестатами 
пользователей со стороны СИ на протяжении всего времени 
жизни аттестатов и соответствующих токенов.

В подразделах~\ref{AUTH.Reg}~--- \ref{AUTH.Cred} процедуры 
детализируются и анализируются. Анализ включает определение 
возможных атак и принципов защиты от них.
\fi