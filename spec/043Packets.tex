\section{Пакеты}\label{COMMON.Packets}

Соответствие тому или иному уровню гарантий реализуется через выполнение 
требований безопасности, определенных в настоящем стандарте.

Требования безопасности схожего назначения группируются в пакеты. 
Пакет~--- это также процедуры, процессы и элементы, которых касаются требования.
%
Перечень пакетов для каждого из блоков представлен в таблице~\ref{Table.COMMON.Packages}.

\begin{table}[hbt]
\caption{Пакеты}\label{Table.COMMON.Packages}
\begin{tabular}{|l|l|c|c|}
\hline
Блок & \multicolumn{1}{c|}{Пакет} & Код & Ссылка\\
\hline
\hline
\multirow{2}{*}{Идентификация} 
  & Регистрация пользователей & РП & \ref{UR}\\
  & Подтверждение личности    & ПЛ & \ref{IP}\\
\hline
\hline
\multirow{3}{*}{Аутентификация} 
  & Выпуск токенов            & ВТ & \ref{TI}\\
  & Управление аттестатами    & УА & \ref{CM}\\
  & Протоколы аутентификации  & ПА & \ref{AP}\\
\hline
\hline
\multirow{3}{*}{Федерация} 
  & Управление федерацией     & УФ & \ref{FM}\\
  & Управление билетами       & УБ & \ref{TM}\\
  & Управление сеансами       & УС & \ref{SM}\\
\hline
\end{tabular}
\end{table}

В преамбуле пакета определяется его назначение, дается обзор понятий, 
необходимых для формулировки требований пакета и правильной их интерпретации. 

Требования внутри пакета нумеруются последовательно, начиная с единицы. Номер
требования через точку присоединяется к коду пакета, в результате получается полное
имя требования: РП.1, УБ.2 и т.~д.

Для каждого требования в круглых скобках перечисляются уровни, на которых 
требование выдвигается, например: (3), (1–3), (2, 3). 

