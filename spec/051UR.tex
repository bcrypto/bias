% User Registration
\section{Регистрация пользователей (РП)}\label{UR}

\subsection{Обзор}\label{UR.Intro}

Пакет РП устанавливает требования по организации регистрации пользователей,
сбору их идентификационных данных.

Регистрация может проходить тремя способами:
\begin{enumerate}
\item
{\it Удаленная регистрация}. 
Выполняется по сетям связи без контроля пользователя со стороны операторов РЦ 
или СИ.
\item
{\it Виртуальная регистрация}. 
Проходит в рамках видеосеанса между пользователем и оператором РЦ. 
\item
{\it Личная регистрация}. 
Проходит при непосредственном (физическом) взаимодействии пользователя и 
оператора РЦ.
\end{enumerate}

РЦ формирует идентификационные данные пользователя или принимает данные, 
сформированные самим пользователем. Затем подтверждается личность пользователя
(см.~\ref{IP}), в том числе проверяется корректность собранных данных. В случае 
успеха РЦ отправляет идентификационные данные СИ. СИ может передать их далее СР.
%
При удаленной регистрации (она допускается только на уровне 1) сбор 
идентификационных данных может напрямую выполнять СИ.

Вместе с идентификационными данными РЦ может собирать биометрические.
Биометрические данные позволяют удостовериться в совпадении пользователей при 
повторной регистрации или, наоборот, выявлять \addendum{не}совпадения. 
%
Кроме этого, биометрические данные выступают в роли свидетельства участия в 
регистрации и препятствуют отказу от регистрации. 
%
Собранные биометрические данные, как и идентификационные, отправляются СИ и 
могут далее пересылаться СР.

Еще одним инструментом неотказуемости является согласие пользователя на 
регистрацию в виде подписанного заявления.

\subsection{Требования}\label{UR.Reqs}

% документирование

%\req{РП}{2, 3}
%Процессы сбора идентификационных данных должны быть документированы в 
%инфраструктуре аутентификации и доведены до сведения РЦ и СИ.

% способ регистрации

\req{РП}{2}
Должна проводиться виртуальная или личная регистрация пользователей.

\req{РП}{3}
Должна проводиться личная регистрация пользователей.

\req{РП}{1}
Во время удаленной регистрации соединение между РЦ и пользователем должно быть 
защищено (с аутентификацией РЦ).

\req{РП}{1, 2}
\addendum{Во время виртуальной регистрации} пользователь не должен покидать свое
место регистрации (там, где установлены видеокамера). В регистрации со стороны
РЦ должен лично участвовать оператор, и он также не должен покидать свое место
регистрации.
%
Оператор должен иметь возможность наблюдать все действия пользователя.
%
Для проверки цифровых компонентов удостоверения место регистрации пользователя
должно быть снабжено сканерами, карт-приемниками, другим необходимым
оборудованием.
%
Соединение между РЦ и пользователем должно быть защищено (с аутентификацией РЦ).

% The CSP SHALL employ physical tamper detection and resistance features
% appropriate for the environment in which it is located. For example, a kiosk
% located in a restricted area or one where it is monitored by a trusted
% individual requires less tamper detection than one that is located in a
% semi-public area such as a shopping mall concourse.

\req{РП}{1, 2}
Операторы, которые проводят виртуальную регистрацию, должны проходить обучение, 
направленное на овладение навыками регистрации, на обнаружение мошенничества со 
стороны регистрируемых пользователей.
%
% The CSP SHALL require operators to have undergone a training program to detect
% potential fraud and to properly perform a supervised remote proofing session.

\req{РП}{3}
Пользователь должен представить РЦ заявление о регистрации, подписанное 
собственноручно.

% сбор данных

\req{РП}{2, 3}
Во время регистрации пользователя должен проводиться сбор его идентификационных
данных. Собираемые данные должны однозначно характеризовать пользователя
в инфраструктуре.

\begin{note}
Обычный достаточный набор идентификационных данных~--- это 
полное имя пользователя, дата и место рождения.
\end{note}

\begin{note}
При сборе идентификационных данных следует придерживаться следующих правил: 

--~не использовать идентификационные данные для проверки полномочий 
и статуса пользователя, для предоставления ему преференций;

--~собирать только необходимые идентификационные данные;

%\item
%фиксировать личный номер только тогда, когда личность пользователя 
%без него установить нельзя;
%
% SP800-63-3A: Overreliance on the SSN can contribute to misuse and place the 
% applicant at risk of harm, such as through identity theft.

--~информировать пользователя о том, как собираемые данные будут использоваться;

--~фиксировать претензии пользователей о недостатках при сборе данных.
\end{note}

% биометрика

\req{РП}{3}
Вместе с идентификационными данными пользователя должны собираться  
биометрические. 

\req{РП}{1--3}
Идентификационные и биометрические данные должны передаваться между
РЦ, СИ и СР по защищенным соединениям (с взаимной аутентификацией сторон).

\req{РП}{1--3}
Должны быть разработаны и реализованы правила хранения идентификационных 
и биометрических данных у РЦ, СИ и СР.
%
Должна обеспечиваться конфиденциальность и контролироваться целостность
хранимых данных, должны быть определены сроки хранения. 
%
При выводе РЦ, СИ и СР из обращения хранимые данные должны уничтожаться.

% schedule of retention

%\req{РП}{1--3}
%При хранении идентификационных и биометрических данных у РЦ, СИ и СР 
%должна обеспечиваться их конфиденциальность и контролироваться целостность.

% аудит

\req{РП}{2,~3}\label{R.UR.Audit}
РЦ и СИ должны вести аудит событий регистрации. В записях аудита 
должны быть указаны представленные при регистрации удостоверения,
виды собранных биометрических данных.

% todo: Записи аудита не обязаны включать фотокопии удостоверений,
% видеозаписи виртуальной регистрации?

