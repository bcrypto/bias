% Identity Proofing
\section{Подтверждение личности (ПЛ)}\label{IP}

\subsection{Обзор}\label{IP.Intro}

Пакет РП устанавливает требования по подтверждению личности регистрируемых 
пользователей.

При подтверждении личности РЦ (через своего оператора) убеждается 
в том, что регистрируемый пользователь действительно существует, 
действительно характеризуется заявленными идентификационными данными
и что эти данные корректны и подлинны.
%
%Дополнительно РЦ может потребоваться организовать подтверждение
%так, чтобы зарегистрированный пользователь не смог впоследствии опротестовать 
%факт регистрации. 

Подтверждение личности может быть выполнено за рамками инфраструктуры
аутентификации. Например, если аттестатом является сертификат открытого ключа, 
выпущенный в доверенной ИОК, то РЦ и СИ могут полагаться на подтверждение 
личности, выполненную при регистрации пользователя в ИОК.
%
Требования по подтверждению в этом случае переносятся на внешние РЦ.

При подтверждении личности РЦ использует одно или несколько удостоверений
пользователя.  
%
Удостоверение~--- это физический (бумага, пластик) документ, выпущенный 
доверенной стороной и содержащий идентификационные данные 
пользователя.
%
%Удостоверение может состоять из нескольких элементов.
%Например, элементы паспорта~--- это его информативные страницы. 
%
При удаленной или виртуальной регистрации пользователь может 
предъявлять не само удостоверение, а его фотокопию.

Выделяются следующие классы удостоверений.

\begin{enumerate}
\item
{\it Слабое удостоверение}. 
При выпуске удостоверения не проводилось подтверждение личности.
Но выпуск организован так, что с высокой достоверностью
удостоверение действительно принадлежит предъявителю. 
Удостоверение содержит номер, который однозначно указывает на владельца
или идентифицирует само удостоверение.
%
Примеры слабых удостоверений~--- 
\addendum{свидетельство о рождении},
карточка сотрудника компании,
диплом об образовании,
\addendum{кредитная карта}.

% At least one reference number that uniquely identifies itself or the person to 
% whom it relates

\item
{\it Адекватное удостоверение}. 
При выпуске удостоверения проводилось подтверждение личности.
Выпуск организован так, что с высокой достоверностью
удостоверение действительно принадлежит предъявителю. 
%
Удостоверение содержит либо номер, который однозначно указывает на 
владельца, либо фотографию, \addendum{либо биометрические} данные владельца.
% 
Если удостоверение содержит цифровые элементы, то их целостность и 
подлинность контролируются криптографическими методами. 
%
Если используются физические элементы защиты (водяные знаки),  
то их воспроизводство или обход требует от противника применения 
дорогостоящих или недоступных технологий. 
%
Примеры адекватных удостоверений~--- \addendum{студенческий билет},
военный билет.

\item
{\it Сильное удостоверение}. 
Адекватное удостоверение, которое выпущено государственным органом с 
соблюдением утвержденных процедур. Удостоверение обязательно содержит 
и номер, и фотографию, а возможно \addendum{и биометрические} данные владельца.
В удостоверении указано официальное имя владельца.
%
Примеры сильных удостоверений~--- паспорт, водительские права, ID-карта.
\end{enumerate}

Удостоверение (включая представленные в нем идентификационные данные) 
проверяется одним из четырех способов:
\begin{enumerate}
\item[а)]
обращение к стороне, выдавшей удостоверение (прямой запрос, 
автоматизированные информационные системы);
%
% ОАИС, регистр населения
%
\item[б)]
контроль элементов удостоверения с помощью технологического оборудования 
и/или программного обеспечения (фотоспектральные сканеры, компьютерное 
зрение);
%
% UK Guidance: confirm the UV and IR security features are genuine
%
\item[в)]
ручная проверка обученными операторами (визуальный и тактильный осмотр 
элементов защиты, сличение фотографии);
%
% check the visible security features are genuine

\item[г)]
проверка криптографических контрольных характеристик
(проверка электронной цифровой подписи).
%
% confirm th UV and IR security features are genuine
%
\end{enumerate}

\addendum{При выборе способа проверки следует учитывать, что при 
удаленной или виртуальной регистрации применение способов б)~--- г)
затруднено или даже невозможно.}

РЦ может проверять не только содержимое удостоверения, но и факт владения 
им. Например, для подтверждения владения кредитной картой пользователю 
может быть предложено оплатить с ее помощью услугу регистрации и 
получить квитанцию с секретным кодом проверки. Этот код пользователь 
предъявляет при регистрации.
%
Похожим образом РЦ проверяет владение физическим (почтовым) 
и электронными (номер мобильного телефона, электронная почта) адресами.

Для повышения гарантий подтверждения личности РЦ дополнительно к 
удостоверениям может использовать записи актов гражданского состояния, 
данные геолокации, характеристики используемых пользователем устройств, 
физические особенности пользователя и др.
%
% The CSP SHOULD obtain additional confidence in identity proofing using fraud 
% mitigation measures (e.g., inspecting geolocation, examining the device 
% characteristics of the applicant, evaluating behavioral characteristics, 
% checking vital statistic repositories such as the Death Master File [DMF], so 
% long as any additional mitigations do not substitute for the mandatory 
% requirements contained herein.

% todo: liveness detection

\subsection{Требования}\label{IP.Reqs}

% удостоверения

\req{ПЛ}{2,~3}
Пользователь должен предъявить РЦ удостоверения, в которых 
представлены все его идентификационные данные, подлежащие сбору в 
ходе регистрации. 
%
РЦ должен проверить удостоверения и, в случае успеха, зафиксировать
их реквизиты вместе с собираемыми идентификационными данными.

%Пользователь должен представить удостоверения, в которых указаны 
%все его идентификационные данные, собираемые в ходе регистрации. 
%
%РЦ должен зафиксировать эти данные вместе с реквизитами удостоверений. 

\req{ПЛ}{2}
Пользователь должен предъявить, как минимум, либо 1 сильное удостоверение,
либо 2 адекватных.

%Пользователь должен представить, по крайней мере, 
%2 элемента сильного удостоверения или 1 элемент сильного и 2 элемента 
%адекватного. РЦ должен проверить представленные элементы.
%РЦ должен проверить все адреса пользователя, указанные при регистрации.

\req{ПЛ}{3}
Пользователь должен предъявить, как минимум, 1 сильное удостоверение.

%\req{ПЛ}{3}
%Пользователь должен представить, по крайней мере, 
%2 элемента сильного удостоверения и 1 элемент адекватного. РЦ должен 
%проверить представленные элементы.  
%РЦ должен проверить все адреса пользователя, указанные при регистрации.

% что проверять

\req{ПЛ}{1--3}
Проверка удостоверения должна покрывать:
\begin{itemize}
\item
атрибуты удостоверения как физического документа 
(если вместо него не предъявлена фотокопия);
\item
представленные в удостоверении идентификационные данные владельца;
\item
реквизиты удостоверения;
\item
срок действия удостоверения.
\end{itemize}

% как проверять

\req{ПЛ}{1--3}
Слабое удостоверение должно проверяться способом~а).

%Элементы слабого удостоверения должны проверяться через 
%обращение к стороне, выдавшей удостоверение.

\req{ПЛ}{1--3}
Адекватное удостоверение должно проверяться одним \addendum{или} несколькими 
способами из списка а)~--~г).

%Элементы адекватного удостоверения должны проверяться через 
%обращение к стороне, выдавшей удостоверение,
%или через контроль физических элементов защиты
%или с помощью обученных операторов (проверка фотографии)
%или путем проверки криптографических контрольных характеристик
%(электронной цифровой подписи).

\req{ПЛ}{1, 2}
Сильное удостоверение должно проверяться одним \addendum{или} несколькими 
способами из списка а)~--~г).

\req{ПЛ}{3}                                                                  
Сильное удостоверение должны проверяться способом~а), 
а также одним или несколькими способами из списка б)~--~г).

%Элементы сильного удостоверения должны проверяться через 
%обращение к стороне, выдавшей удостоверение,
%а также дополнительно через контроль физических элементов защиты 
%или с помощью обученных операторов или путем проверки криптографических 
%контрольных характеристик.

% дополнительные проверки

\req{ПЛ}{1--3}
Если физический или электронный адрес пользователя не проверен
как элемент удостоверения, то РЦ должен отправить по этому 
адресу секретный код и этот код пользователь должен предъявить 
для завершения регистрации. 
%
Код для проверки электронного адреса должен действовать не более суток.
%
Код для проверки физического адреса должен действовать не более 21 дня.
%
Код должен содержать не менее 32 битов энтропии.

