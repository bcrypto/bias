\newpage
\setcounter{page}{1}
\pagestyle{headings}

\begin{center}
{\bfseries
ГОСУДАРСТВЕННЫЙ СТАНДАРТ РЕСПУБЛИКИ~БЕЛАРУСЬ
\vskip 2pt
\hrule width\textwidth

\vskip 9pt

Информационные технологии и безопасность

ИНФРАСТРУКТУРЫ АУТЕНТИФИКАЦИИ

\vskip 9pt

Iнформацыйныя тэхналогii i бяспека

ІНФРАСТРУКТУРЫ АЎТЭНТЫФІКАЦЫІ
}
% bfseries

\vskip 9pt

Information technology and security

Authentication frameworks

\vskip 4pt                
\hrule width \textwidth
\end{center}

\mbox{}\hfill{\bfseries Дата введения 202X-XX-XX}

\chapter{Область применения}\label{Scope}

Настоящий стандарт устанавливает правила построения инфраструктур 
аутентификации и требования к блокам инфраструктур. 
%        
Охватываются вопросы идентификации пользователей, в том числе их регистрации и
подтверждения личности, организации аутентификации, распространения сведений об
аутентификации среди сторон, связанных отношениями доверия.
%
Представлена технология OpenID Connect (OIDC), ориентированная на построение
в Интернете крупных открытых инфраструктур аутентификации.

Настоящий стандарт применяется при разработке информационных систем,
в которых используются <<цифровые образы>> пользователей, а также 
сопровождающих эти системы средств аутентификации и средств криптографической 
защиты информации.

