\section{Клиентская программа}\label{COMMON.CP}

Пользователь взаимодействует с ПС и СИ с помощью КП.
КП, как правило, выполняется на персональном компьютере или мобильном 
устройстве пользователя. Это может быть браузер, отдельное приложение или 
связка браузера с приложением.  

КП может обрабатывать критические данные
(например, секреты аутентификации), а также открытые данные, 
целостность и подлинность которых определяют надежность  
взаимодействия с ПС и СИ (например, запросы аутентификации).

КП не может обеспечить полную защиту обрабатываемых данных. Защита организуется
средствами системной среды, в том числе через настройки операционной системы.
%
Основные задачи защиты: 
невозможность чтения критических областей памяти вредоносными программами;
невозможность перехвата данных, передаваемым по каналам управления; 
защита канала <<браузер~--- приложение>>; 
невозможность подмены открытых данных.
%
Организация защиты выходит за рамки настоящего стандарта.

КП может быть выполнена в виде СКЗИ в соответствии с требованиями 
СТБ~34.101.27.

В ходе взаимодействия КП~-- СИ выполняется аутентификация пользователя.
При аутентификации КП использует ТА пользователя. 
%
При использовании в качестве ТА аппаратного КТ единственными критическими
данными, которые обрабатывает КП, является пароль доступа к КТ. Поэтому КТ
желательно использовать в тех ситуациях, когда КП пользователя выполняется в
потенциально агрессивной среде, например, на незащищенном компьютере общего
пользования.
