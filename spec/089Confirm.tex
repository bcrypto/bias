\section{Подтверждение операций}\label{OIDC.Confirm}

{\bf Билет и запрос подтверждения}.
%
При совершении определенных цифровых операций ПС может требоваться
согласие пользователя, прошедшего аутентификацию и получившего БА. 
Согласие оформляется билетом подтверждения. Этот билет представляет собой 
объект JWS (см.~\ref{JWT.JWS}), тело которого содержит следующие поля: 
\begin{itemize}
\item[--] 
{\bf \lstinline{client_id}}~--- идентификатор ПС;

\item[--] 
{\bf \lstinline{user_id}}~--- идентификатор пользователя;

\item[--] 
{\bf \lstinline{id_token_hash}}~--- хэш-значение БА;

\item[--] 
{\bf \lstinline{attrs_hash}}~--- хэш-значение объекта JSON,
описывающего атрибуты операции; 

\item[--] 
{\bf \lstinline{confirmer_id}}~--- идентификатор
подтверждающей (подписывающей) стороны: пользователя или СИ;
\end{itemize}

В заголовке билета указывается алгоритм подписи (поле~\lstinline{alg})
и цепочка сертификатов, соответствующих личному ключу подписывающего.
Подпись указывается в третьем компоненте объекта JWS.
%
Включение в билет хэш-значения БА подтверждает подлинность 
пользователя, участвующего в операции.
%
Формат объекта, описывающего атрибуты операции, в настоящем стандарте не 
уточняется.

Если билет подписывает сам пользователь, 
то подпись прямо свидетельствует об его согласии с условиями
операции и не позволяет отказаться от согласия впоследствии.
%
Однако пользователь может не располагать средством ЭЦП, или его 
открытый ключ может быть не зарегистрирован в инфраструктуре,
признаваемой ПС. В таких случаях билет подписывает СИ,
которому доверяют и пользователь, и ПС. 

Билет подтверждения формируется на основании запроса.
Запрос~--- это содержимое билета без заголовка и подписи. 

Запрос и билет передаются в теле
HTTP-пакетов, тип содержимого \lstinline{Content-Type}
имеет значение \lstinline{application/json}.
Запрос подтверждения может передаваться
в HTTP-ответе с помощью механизма перенаправления (redirect) и
в HTTP-запросе, при этом параметры запроса могут
передаваться как query-параметры в URL перенаправления и запроса соответственно
(т.~е. также, как параметры запроса аутентификации).

{\bf Подтверждение с помощью сетевого токена}.
Для подтверждения операции пользователем, не располагающим средством 
ЭЦП, может использоваться сетевой ТА. На этот токен СИ присылает секретный 
код подтверждения. Ввод пользователем кода на странице ПС означает его 
согласие с условиями операции.

Подтверждение с помощью сетевого токена выполняется следующим образом: 
\begin{enumerate}
\item 
КП запрашивает у ПС операцию.

\item 
ПС отправляет КП атрибуты операции и запрос на ввод кода подтверждения.
КП вычисляет хэш-значение \lstinline{attrs_hash}.

\item 
ПС готовит и отправляет СИ запрос подтверждения.
В поле \lstinline{confirmer_id} запроса указывается идентификатор СИ. 
Запрос сопровождается БА.

\item 
СИ проверяет БА и сравнивает его хэш-значение со 
значением~\lstinline{id_token_hash}, указанным в запросе.
%
В случае успеха СИ возвращает ПС идентификатор выполняемой операции.
Идентификатор необходим для ссылки на операцию
при последующем обращении ПС к СИ.
Идентификатор может представлять собой краткосрочный куки.

\item 
СИ генерирует случайный код подтверждения.
%
СИ пересылает код вместе с хэш-значением \lstinline{attrs_hash}
на сетевой токен пользователя.

\item 
Пользователь получает код подтверждения от СИ
и сравнивает полученное хэш-значение \lstinline{attrs_hash} 
с ранее вычисленным. 

\item 
Пользователь вводит код в КП (например, в html-форме на странице ПС).
КП повторяет запрос к ПС, но уже с кодом подтверждения.

\item 
ПС отправляет СИ код подтверждения вместе с идентификатором текущей операции.

\item 
СИ проверяет код, подписывает билет подтверждения и пересылает его ПС.

\item
ПС завершает операцию.

\item 
ПС отправляет КП ответ с результатом выполнения операции.
Это HTTP-ответ на HTTP-запрос к ПС на шаге 7.
\end{enumerate}

