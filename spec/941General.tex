\hiddensection{Общие сведения}\label{CLAIMS.General}

БА и ответ с узла UserInfo содержат утверждения: об аутентификации и о 
пользователе. Стандартные утверждения аутентификации определяются в~\ref{CLAIMS.Auth}, 
стандартные утверждения о пользователе~--- в~\ref{CLAIMS.User}. 
%
Утверждение \lstinline{sub} содержит идентификатор пользователя, прошедшего 
аутентификацию, и относится одновременно к обеим группам утверждений.

Утверждения передаются в полях объектов JSON. Значение стандартного 
утверждения, если не оговорено противное,~--- это строка JSON.

Перечень утверждений может расширяться. При введении дополнительных утверждений 
требуется определить имена и синтаксис их значений. 
%
Следует минимизировать возможность конфликта имен утверждений, например, 
следуя правилам~\cite{RFC7519} и используя так называемые 
устойчивые к коллизиям (collision-resistant) имена. 

Перечень нестандартных утверждений должен быть предварительно согласован между
ПС и СИ. Нераспознанные утверждения должны игнорироваться.

Перечни утверждений, которые включаются в БА и в ответ с узла UserInfo, 
регулируются ПС в запросе авторизации~/ аутентификации (см.~\ref{CLAIMS.ReqWith}).
%
Запрашиваемые утверждения делятся на существенные и несущественные.
%
Запрашивая утверждение как существенное, ПС информирует о том, что 
утверждение необходимо для выполнения определенных сервисов, связанных с 
пользователем.
%
Несущественное утверждение желательно, но не необходимо.
%
Независимо от того, является ли запрошенное утверждение существенным или 
несущественным, СИ при обработке запроса не должен генерировать ошибку, если 
утверждение не может быть возвращено по причине отсутствия необходимой 
информации или из-за отказа пользователя в передаче данной информации 
(если в описании утверждения не оговорено противное).

