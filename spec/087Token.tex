\section{Узел Token}\label{OIDC.Token}

Узел Token обрабатывает запросы на выпуск и обновление билетов.
%
Узел используется в схемах~Code и Hybrid. 
% 
К узлу обращается ПС, предъявляя ранее полученные секретные данные:
код авторизации или БО.
%
СИ проверяет представленные данные и, в случае успеха, выпускает БД, 
возможно БА, возможно БО.

Запрос на выпуск билета включает код авторизации, полученный КП от узла 
Authorization и перенаправленный ПС на узел Redirection. 
%
Запрос на обновление включает БО, полученный ранее от узла Token в результате 
обработки запроса на выпуск или предыдущего запроса на обновление.
%
Тип запроса указывается в его параметре \lstinline{grant_type}.

Адрес узла должен быть зафиксирован в документах инфраструктуры
или адрес должен быть сообщен ПС во время регистрации в инфраструктуре.
%
Адрес не должен содержать компонент \lstinline{fragment}.

Перед обращением к узлу ПС должна пройти аутентификацию перед СИ. 
Аутентификация может проводиться или завершаться в момент обращения,
и тогда запрос ПС должен содержать аутентификационные данные. 
%
Перечень методов аутентификации должен быть согласован со СИ при 
регистрации ПС и связан с идентификатором ПС.

Запросы на узел и соответствующие ответы описаны в 
таблице~\ref{Table.OIDC.Token}. 
%
Запросы и ответы могут включать параметры, дополнительные к перечисленным в 
таблице.

ПС должна обращаться к узлу используя HTTP-метод POST.
%
Запрос должен кодироваться по схеме Form.
%
Нераспознанные параметры запроса и параметры без значений должны 
игнорироваться.
%
Параметры не должны повторяться.

Ответ узла должен кодироваться по схеме JSON. 
%
Если ответ содержит БД, БО или другие критические данные, то соответствующий 
пакет HTTP должен включать заголовок \lstinline{Cache-Control} со значением
\lstinline{"no-store"} и заголовок \lstinline{Pragma} со значением 
\lstinline{"no-cache"}.

\begin{table}[H]
\caption{Узел Token: запросы и ответы}\label{Table.OIDC.Token}  
\begin{tabular}{|l|c|p{8.1cm}|l|}
\hline
\multirow{2}{*}{Параметр} & Вклю- & \multirow{2}{*}{Описание} & \multirow{2}{*}{Ссылка}\\
                          & чение &&\\
\hline
\hline
\multicolumn{4}{|c|}{Запрос на выпуск билетов (POST/Form)}\\
\hline
\hline
%
\lstinline!grant_type! & + & 
Значение \lstinline!"authorization_code"! & 
\ref{PARAMS.GrantType}\\
\hline
%
\lstinline!code! & + & 
Код авторизации & 
\ref{PARAMS.Code}\\
\hline
%
\lstinline!redirect_uri! & + & 
Копия одноименного параметра запроса на узел Authorization & 
\ref{PARAMS.RedirectUri}\\
\hline
%	
\lstinline!client_id! & у & 
Идентификатор ПС & 
\ref{PARAMS.ClientId}\\
\hline
%	
\lstinline!client_secret! & у & 
Секрет аутентификации ПС$\mbox{}^*$ & 
\ref{PARAMS.ClientSecret}\\
\hline
%
\lstinline!client_assertion_type! & у &
Тип аутентификатора ПС & 
\ref{PARAMS.ClientAssertionType}\\
\hline
%	
\lstinline!client_assertion! & у & 
Аутентификатор ПС & 
\ref{PARAMS.ClientAssertion}\\
\hline
%
\hline
\multicolumn{4}{|c|}{Запрос на обновление билетов (POST/Form)}\\
\hline
\hline
%
\lstinline!grant_type! & + & 
Значение \lstinline!"refresh_token"! & 
\ref{PARAMS.GrantType}\\
\hline
%
\lstinline!refresh_token! & + &
БО & 
\ref{PARAMS.RefreshToken}\\
\hline
%	
\lstinline!client_id! & у & 
Идентификатор ПС & 
\ref{PARAMS.ClientId}\\
\hline
%	
\lstinline!client_secret! & у & 
Секрет аутентификации ПС$\mbox{}^*$ & 
\ref{PARAMS.ClientSecret}\\
\hline
%
\lstinline!client_assertion_type! & у &
Тип аутентификатора ПС & 
\ref{PARAMS.ClientAssertionType}\\
\hline
%	
\lstinline!client_assertion! & у & 
Аутентификатор ПС & 
\ref{PARAMS.ClientAssertion}\\
\hline
%	
\lstinline!scope! & о & 
Запрашиваемая область действия & 
\ref{PARAMS.Scope}\\
\hline
%
\hline
\multicolumn{4}{|c|}{Успешный ответ на запрос на выпуск билетов (JSON)}\\
\hline
\hline
%
\lstinline!access_token! & + &
БД & 
\ref{PARAMS.AccessToken}\\
\hline
%
\lstinline!token_type! & + & 
Тип БД & 
\ref{PARAMS.TokenType}\\
\hline
%
\lstinline!expires_in! & р & 
Срок действия БД & 
\ref{PARAMS.ExpiresIn}\\
\hline
%
\lstinline!id_token! & + &
БА & 
\ref{PARAMS.IdToken}\\
\hline
%
\lstinline!refresh_token! & у &
БО & 
\ref{PARAMS.RefreshToken}\\
\hline
%	
\lstinline!scope! & у & 
Разрешенная область действия & 
\ref{PARAMS.Scope}\\
\hline
%
\hline
\multicolumn{4}{|c|}{Успешный ответ на запрос на обновление билетов (JSON)}\\
\hline
\hline
%
\lstinline!access_token! & + & 
Билет доступа & 
\ref{PARAMS.AccessToken}\\
\hline
%
\lstinline!token_type! & + &
Тип билета доступа & 
\ref{PARAMS.TokenType}\\
\hline
%
\lstinline!expires_in! & р & 
Срок действия БД & 
\ref{PARAMS.ExpiresIn}\\
\hline
%
\lstinline!id_token! & о &
БА & 
\ref{PARAMS.IdToken}\\
\hline
%
\lstinline!refresh_token! & у &
БО & 
\ref{PARAMS.RefreshToken}\\
\hline
%	
\lstinline!scope! & у & 
Разрешенная область действия & 
\ref{PARAMS.Scope}\\
\hline
%
\hline
\multicolumn{4}{|c|}{Ответ об ошибке (JSON)}\\
\hline
\hline
%
\lstinline!error! & $+$ & 
Код ошибки & 
\ref{PARAMS.Error}\\
\hline
%
\lstinline!error_description! & о & 
Описание ошибки &
\ref{PARAMS.ErrorDescr}\\
\hline
%
\lstinline!error_uri! & о & 
Адрес веб-страницы с информацией об ошибке &
\ref{PARAMS.ErrorUri}\\
\hline
\multicolumn{4}{p{0.95\linewidth}}{$\mbox{}^*$~--- может передаваться 
отдельно по правилам схемы аутентификации HTTP Basic.}
\end{tabular}
\end{table}

Для информирования СИ о необходимости выпуска БО в параметр \lstinline{scope} 
запроса на узел Authorization (см. таблицу~\ref{Table.OIDC.Authorization}) 
включается строка \lstinline{"offline_access"}. 
%
Для информирования о необходимости перевыпуска БО эта же строка включается в 
параметр \lstinline{scope} запроса на обновление.
%
СИ может выпускать и перевыпускать БО в других ситуациях, без явного 
запроса со стороны ПС.

В запросах на узел Token параметры \lstinline{client_id}, 
\lstinline{client_secret}, \lstinline{client_assertion_type} и 
\lstinline{client_assertion} используются для аутентификации ПС в момент 
обращения.
%
Предусмотрены 4 метода аутентификации.

\begin{enumerate}
\item
Метод \lstinline{client_secret_basic}.
%
При регистрации ПС согласует со СИ секрет аутентификации 
\lstinline{client_secret}, СИ связывает секрет с идентификатором ПС 
\lstinline{client_id}.
%
Секрет передается в качестве аутентификатора ПС в запросе на узел.
Секрет сопровождается идентификатором ПС.
%
СИ определяет по идентификатору зарегистрированный секрет, а затем сравнивает  
его с присланным. 
%
Секрет \lstinline{client_secret} передается по правилам схемы аутентификации 
HTTP Basic (в заголовке \lstinline{Authorization}).

\item
Метод \lstinline{client_secret_post}.
%
Отличается от метода \lstinline{client_secret_basic} только тем, что 
\lstinline{client_secret} кодируется вместе с остальными параметрами 
запроса по схеме Form.

\item
Метод \lstinline{client_secret_jwt}.
%
При регистрации ПС согласует со СИ секретный ключ имитозащиты
\lstinline{client_secret}.
%
В качестве аутентификатора передается объект JWT, который включает 
идентификаторы ПС и СИ, уникальный идентификатор самого объекта, отметки 
времени, а также имитовставку всех этих данных, вычисленную на ключе 
\lstinline{client_secret}.
%
СИ определяет по идентификатору зарегистрированный ключ и проверяет на нем 
имитовставку.  
%
Объект JWT передается в параметре \lstinline{client_assertion} запроса.
%
Метод аутентификации указывается в параметре \lstinline{client_assertion_type}.

\item
Метод \lstinline{private_key_jwt}.
%
Отличается от метода \lstinline{client_secret_jwt} тем, что вместо имитовставки 
используется ЭЦП. 
%
Данные объекта JWT подписываются на личном ключе ПС. СИ проверяет подпись 
на соответствующем открытом ключе. Сертификат открытого ключа передается СИ в 
момент регистрации ПС.
\end{enumerate}

