\section{Уровни гарантий}\label{COMMON.Levels}

В настоящем стандарте каждый из блоков <<идентификация>>, <<аутентификация>>, 
<<федерация>> удовлетворяет определенному уровню гарантий: базовому (1), 
среднему (2) или высокому (3).
%
Уровень гарантий определяет степень уверенности в соблюдении процедур и правил, 
за которые отвечает блок.
%
Уровни кратко охарактеризованы в таблице~\ref{Table.COMMON.AL}. 

\begin{table}[hbt]
\caption{Уровни гарантий}\label{Table.COMMON.AL}
\begin{tabular}{|c|p{14cm}|}
\hline
Уровень & Краткое описание\\
\hline
\hline
\multicolumn{2}{|l|}{Идентификация}\\
\hline
\hline
%
1 & Без подтверждения личности. При регистрации пользователь может
указать идентификационные атрибуты, корректность которых подтверждается 
только им самим.\\ 
\hline
%
2 & Подтверждение личности через контроль удостоверений.
Регистрация проходит виртуально (в рамках видеосеанса) или в личном присутствии 
пользователя.\\
\hline
%
3 & Подтверждение личности через контроль удостоверений квалифицированным 
персоналом. Регистрация проходит в личном присутствии пользователя.
Сбор биометрических данных.\\
%
\hline
\hline
\multicolumn{2}{|l|}{Аутентификация}\\
\hline
\hline
%
1 & Однофакторная аутентификация.\\
\hline
%
2 & Многофакторная аутентификация.\\
\hline
%
3 & Многофакторная аутентификация с использованием аппаратного устройства 
и КТ.\\
%
\hline
\hline
\multicolumn{2}{|l|}{Федерация}\\
\hline
\hline
%
1 & Распространяемые утверждения подписываются.\\
\hline
%
2 & Распространяемые утверждения подписываются и зашифровываются 
(конвертуются).\\
\hline
%
3 & Распространяемые утверждения подписываются и зашифровываются.
Аутентифицированный пользователь, субъект утверждений, должен доказать владение 
ключом, ссылка на который сопровождает утверждения.\\ 
\hline
\end{tabular}
\end{table}

Уровень гарантий блока следует выбирать с учетом критичности последствий 
возможных ошибок (идентификации, аутентификации или федерации).
%
Рекомендации по выбору уровня гарантий даны в приложении~\ref{AL}.

Уровни гарантий блоков могут отличаться друг от друга. Например, в
информационной системе может применяться средний уровень гарантий идентификации
(проверяются удостоверения регистрируемого пользователя), высокий уровень
гарантий аутентификации (аутентификация с помощью персонального аппаратного КТ)
и базовый уровень гарантий федерации (распространяются подписанные билеты
аутентификации OIDC).

