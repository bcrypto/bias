\begin{appendix}{Е}{рекомендуемое}{Выбор уровня гарантий}\label{AL}

\hiddensection{Ошибки}\label{AL.Errors}

Уровень гарантий идентификации, аутентификации и федерации выбирается 
по результатам анализа последствий ошибок. Ошибки могут быть вызваны атаками.
Перечень основных атак представлен в приложении~\ref{ATK}.

Основная ошибка идентификации~--- это некорректное связывание аттестата (учетной записи)
пользователя с ним самим. Другие ошибки: сбор некорректных 
идентификационных данных, сбор лишних данных.

Основная ошибка аутентификации~--- это некорректный вывод о подлинности 
стороны. Ошибка, как правило, индуцируется атаками противника.

Основная ошибка федерации~--- это распространение некорректных утверждений
об аутентификации пользователя или об его идентификационных данных. Речь идет о 
подделке билетов или их попадании в руки противника. К ошибкам федерации также 
относится нарушение отношений доверия, например, между ПС и СИ. 

\hiddensection{Последствия}\label{AL.Impact}

Типовые категории последствий от ошибок идентификации, аутентификации и 
федерации и степень их критичности представлены в 
таблице~\ref{Table.AL.Impact}. 
%
С помощью таблицы следует определить степень критичности ошибок для каждой из
категорий.

\begin{table}[hbt]
\caption{Последствия ошибок}\label{Table.AL.Impact}
{\small
\begin{tabular}{|p{3.3cm}|p{4cm}|p{4cm}|p{4cm}|}
\hline
\multicolumn{1}{|c|}{Категория} & \multicolumn{3}{c|}{Степень критичности}\\
\cline{2-4}
\multicolumn{1}{|c|}{последствий} & \multicolumn{1}{c|}{Низкая} 
& \multicolumn{1}{c|}{Средняя} 
& \multicolumn{1}{c|}{Высокая}\\
\hline
\hline
Репутационные потери & 
Незначительные краткосрочные & 
Значительные краткосрочные или незначительные долгосрочные &  
Серьезные долгосрочные, потенциально затрагивающие многие стороны\\
%
\hline
Финансовые потери & 
Незначительные &
Значительные &
Серьезные или катастрофические\\
%
\hline
Вред организациям~/ институтам & 
Заметное снижение эффективности & 
Существенное снижение эффективности & 
Потеря работоспособности\\ 
%
\hline
Раскрытие конфиденциальной информации &
Частичное раскрытие, несущественная информация &
Существенная информация &
Критическая информация\\
%
\hline
Личная безопасность &
Легкие травмы, без медицинского вмешательства &
Средний риск легких травм или незначительный риск травм, требующих 
медицинского вмешательства &
Серьезные травмы или смерть\\
%
\hline
Правонарушение & 
Гражданское или уголовное, не требуется вмешательство сил правопорядка &
Гражданское или уголовное, требуется вмешательство сил правопорядка &
Гражданское или уголовное, требуется вмешательство крупных сил правопорядка\\
\hline
\end{tabular}
}
\end{table}

\hiddensection{Выбор уровня}\label{AL.Levels}

Уровень гарантий выбирается с помощью таблицы~\ref{Table.AL.Levels}.
%
По таблице определяется минимальный уровень, для которого степень критичности
последствий ошибок допустима относительно каждой из категорий последствий.
%
Этот уровень считается рекомендуемым. 

\begin{table}[hbt]
\caption{Допустимая степень критичности последствий ошибок}
\label{Table.AL.Levels}
{\small
\begin{tabular}{|l|c|c|c|}
\hline
Категория последствий & \multicolumn{3}{|c|}{Уровень гарантий}\\
\cline{2-4}
& 1 & 2 & 3\\
\hline
\hline
\multirow{3}{*}{Репутационные потери}
& низкая & низкая  & низкая\\
&       & средняя & средняя\\
&       &         & высокая\\
%
\hline
\multirow{3}{*}{Финансовые потери}
& низкая & низкая  & низкая\\
&        & средняя & средняя\\
&        &         & высокая\\
%
\hline
\multirow{3}{*}{Вред организациям~/ институтам}
& без последствий & низкая  & низкая\\ 
&                 & средняя & средняя\\ 
&                 &         & высокая\\ 
%
\hline
\multirow{3}{*}{Раскрытие конфиденциальной информации}
& без последствий & низкая  & низкая\\ 
&                 & средняя & средняя\\ 
&                 &         & высокая\\ 
%
\hline
\multirow{3}{*}{Личная безопасность}
& без последствий & низкая & низкая\\ 
&                 &        & средняя\\ 
&                 &        & высокая\\ 
%
\hline
\multirow{3}{*}{Правонарушение}
& без последствий & низкая  & низкая\\ 
&                 & средняя & средняя\\ 
&                 &         & высокая\\ 
\hline
\end{tabular}
}
\end{table}

При выборе уровня гарантий идентификации анализ в соответствии с 
таблицей~\ref{Table.AL.Levels} не проводится и в качестве рекомендуемого 
сразу назначается уровень 1, если сбор идентификационных данных не 
предполагается или если корректность идентификационных данных, распространяемых 
после аутентификации их владельца, не требуется.

Рекомендуемый уровень 1 гарантий аутентификации меняется на уровень 2,
если выполнено одно из следующих условий:
\begin{itemize}
\item
предполагается сбор идентификационных данных;
\item
планируется открытая публикация идентификационных данных (даже самозаявленных);
\item
уровень гарантий идентификации выше первого.
\end{itemize}

При выборе уровня гарантий федерации анализ в соответствии с 
таблицей~\ref{Table.AL.Levels} не проводится и в качестве рекомендуемого 
сразу назначается уровень 1, если распространение утверждений в федерации не 
планируется.

Рекомендуемый уровень гарантий может усиливаться по желанию 
разработчика инфраструктуры аутентификации.

\end{appendix}
