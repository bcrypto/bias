\chapter{Термины и определения, сокращения}\label{TERMS}

В настоящем стандарте применяют термины, установленные в СТБ 34.101.19, 
СТБ 34.101.23, СТБ 34.101.27, СТБ 34.101.31, СТБ 34.101.45 и~СТБ 34.101.65, 
а также следующие термины с соответствующими определениями. 

{\bf \thedefctr~авторизация}:
Назначение прав доступа.

\begin{note*}
В настоящем стандарте права назначает служба идентификации,
права назначаются прикладной системе, права касаются ресурсов пользователя
(их владельца).
\end{note*}

{\bf \thedefctr~активация}:
Перевод объекта или устройства в оперативное состояние,
которое открывает доступ к другим объектам и устройствам. 

\begin{note*}
Доступ, открываемый при активации одного объекта или устройства,
может касаться активации другого.
%
Например, активация пароля состоит в проверке того, что 
владелец криптографического токена ввел его правильно 
и вследствие этого активировал токен, т.~е. получил доступ к его объектам.  
\end{note*}

{\bf \thedefctr~атрибут (идентификационный)}:
Компонент идентификационных данных.

{\bf \thedefctr~аттестат}:
Информационный объект, который связывает идентификационные 
данные пользователя с его токенами аутентификации,
создается при регистрации пользователя,
используется в процессе аутентификации.

{\bf \thedefctr~аутентификатор}:
Полученные с помощью токена аутентификации данные, 
которые подтверждают владение им или его знание. 

{\bf \thedefctr~аутентификация}:
Проверка подлинности стороны.

% skzi: проверка подлинности стороны или данных.

{\bf \thedefctr~билет}:
Информационный объект, который выпускается по итогам успешной аутентификации
пользователя или после предъявления другого билета и подтверждает определенные  
события, факты или состояние. 

% событие: аутентификация, факт: идентификационные данные, состояние: сеанс. 

{\bf \thedefctr~билет аутентификации}; БА:
Билет, который содержит утверждения аутентификации и, возможно, другие
утверждения.

{\bf \thedefctr~билет доступа}; БД:
Билет, который подтверждает права доступа к определенным ресурсам определенного 
пользователя.

{\bf \thedefctr~билет именной}:
Билет, который содержит идентификационные данные пользователя, прошедшего 
аутентификацию.

{\bf \thedefctr~билет на предъявителя}:
Билет, который не содержит идентификационных данных пользователя, прошедшего 
аутентификацию.

{\bf \thedefctr~билет обновления}; БО:
Билет, при предъявлении которого можно получить новый билет 
аутентификации, доступа или обновления без повторной аутентификации.

% todo: уточнить "аутентификации, доступа или обновления"

{\bf \thedefctr~билет сеанса (куки)}; БС:
Билет, который подтверждает открытие сеанса с другой стороной. 

\begin{note*}
В Интернет-системах билет сеанса обычно разделяется между браузером и 
сервером и называется куки (от англ. cookie).
\end{note*}

{\bf \thedefctr~защищенное соединение}: % bpki, btok
Соединение, которое обеспечивает конфиденциальность, 
контроль целостности и, возможно, подлинности сообщений. 

\begin{note*}
Контроль подлинности сообщений от стороны~$A$ к стороне~$B$ 
обеспечивается после того, как~$B$ провела аутентификацию~$A$.
\end{note*}

{\bf \thedefctr~идентификатор}:
Данные, ассоциированные с определенной стороной и позволяющие 
отличить ее от других сторон.

% Unique data used to represent a person’s identity and associated attributes.  

% A bit string that is associated with a person, device or organization.

% A unique label used by a system to indicate a specific entity, object, or 
% group.

% Information used to claim an identity, before a potential corroboration by a  
% corresponding authenticator

{\bf \thedefctr~идентификационные данные}:
Описание характеристик идентичности. 

{\bf \thedefctr~идентификация}: % skzi
Назначение уникального идентификатора или сравнение предъявляемого 
идентификатора с назначенными идентификаторами.

\begin{note*}
В настоящем стандарте под идентификацией в основном понимается назначение
идентификаторов с предшествующими регистрацией и подтверждением личности.
\end{note*}

{\bf \thedefctr~идентичность}:
Набор характеристик, который однозначно описывает определенную сторону в 
определенном контексте.

% An attribute or set of attributes that uniquely describe a subject within a 
% given context.

% Information that is unique within a security domain and which is recognized as 
% denoting a particular entity within that domain.

% The set of physical and behavioral characteristics by which an individual is 
% uniquely recognizable.

%\begin{note*}
%В разных контекстах могут использоваться различные характеристики одной и той 
%же стороны.
%\end{note*}

%{\bf \thedefctr~имя}:
%Вид идентификационных данных, строка сравнительно небольшой длины
%в естественном алфавите, которой удобно управлять человеку.

{\bf \thedefctr~инфраструктура аутентификации}:
Совокупность сторон, которые реализуют сервисы единой аутентификации и 
авторизации, централизованного управления ресурсами пользователей,
а также сторон, которые используют эти сервисы.

{\bf \thedefctr~клиентская программа}; КП:
Программа, которая организует взаимодействие между пользователем,
его токенами аутентификации, прикладной системой и службой идентификации.

% bpki: 

{\bf \thedefctr~код авторизации (OIDC)}:
Ссылочный билет.

{\bf \thedefctr~коммуникационная схема (OIDC)}:
Схема взаимодействия сторон, в том числе описание узлов,
последовательность пересылаемых сообщений и общее 
содержание сообщений.

{\bf \thedefctr~конвертованные данные}: % bpki
Данные, защищенные на секретном ключе и сопровождаемые этим ключом, 
защищенным в свою очередь на открытом ключе получателя.

{\bf \thedefctr~криптографический токен}; КТ:
Аппаратный или программный токен аутентификации, секретом которого 
является личный или секретный криптографический ключ, а аутентификатором~---  
электронная цифровая подпись, имитовставка или их производные. 

\begin{note*}
На криптографическом токене могут размещаться идентификационные данные 
владельца.
\end{note*}

{\bf \thedefctr~одноразовый пароль, OTP}: % brng
Пароль, действие которого ограничено сеансом аутентификации или промежутком 
времени.

\begin{note*}
Аббревиатура OTP~--- от англ. One-Time Password.
\end{note*}

{\bf \thedefctr~подписанные данные}: % bpki
Данные, сопровождаемые электронной цифровой подписью отправителя. 

{\bf \thedefctr~подтверждение личности}:
Проверка того, что пользователь с заявленными идентификационными данными
действительно существует, данные касаются именно этого пользователя и однозначно
его характеризуют.

% Сбор, валидация и верификация идентификационных данных.

% \begin{note*}
% При валидации проверяется корректность данных,
% например, определенного номера телефона или почтового адреса.
%
% При верификации проверяется подлинность данных,
% например, что регистрируемый пользователь действительно владеет определенным 
% номером телефона или проживает по определенному адресу.
% \end{note*}

% NIST SP 800-63-3: The process by which a CSP collects, validates, and 
% verifies information about a person.

% Verifying the claimed identity of an applicant by authenticating the identity 
% source documents provided by the applicant.

% The process of providing sufficient information (e.g., identity history, 
% credentials, documents) to establish an identity. 

% The process by which a CSP or Registration Authority (RA) collect, validate 
% and verify information about a person for the purpose of issuing credentials to 
% that person.

{\bf \thedefctr~пользователь}:
Физическое лицо, регистрируемое или зарегистрированное в инфраструктуре 
аутентификации для доступа к сервисам аутентификации и авторизации.
                
% NIST SP 800-63-3/Subscriber: A party who has received a credential or 
% authenticator from a CSP. 

{\bf \thedefctr~прикладная система}; ПС:
Сторона, которая использует сервисы аутентификации и авторизации,
предоставляемые инфраструктурой аутентификации.

\begin{note*}
Прикладная система инициирует аутентификацию пользователя 
и свою авторизацию на доступ к его ресурсам.
%
Прикладная система имеет собственные ресурсы-услуги, 
которые могут быть предоставлены пользователям после их аутентификации.
\end{note*}

{\bf \thedefctr~регистрационный центр}; РЦ:
Сторона, которая формирует идентификационные данные пользователя
или проверяет и заверяет их для службы идентификации. 

%\begin{note*}
%Регистрационный центр может входить в состав службы идентификации 
%или быть отделен, но связан с нeй.
%\end{note*}

{\bf \thedefctr~ресурс}:
Данные или сервис определенной стороны. 

%\begin{note*}
%Ресурсы пользователя включают в себя его идентификационные данные,
%ресурсы прикладной системы~--- оказываемые ею цифровые услуги.
%\end{note*}

{\bf \thedefctr~сеанс}:
Логическая связь между двумя сторонами, которая описывается 
идентификатором, параметрами защиты и другими согласованными между 
сторонами данными, которые могут быть использованы в нескольких 
соединениях. 

{\bf \thedefctr~секрет аутентификации}:
Секретные данные, которые содержатся в токене аутентификации
и используются для построения аутентификаторов.

{\bf \thedefctr~сервер ресурсов}; СР:
Сторона, которой пользователи делегировали управление своими ресурсами и которая
открывает доступ к этим ресурсам другим сторонам после авторизации.

{\bf \thedefctr~служба идентификации}; СИ:
Сторона, которая проводит идентификацию и аутентификацию пользователей 
и авторизует доступ к их ресурсам.
	
{\bf \thedefctr~соединение}:
Непостоянный канал связи между двумя сторонами. 

{\bf \thedefctr~ссылочный билет}:
Вспомогательный одноразовый билет, который ссылается на другой билет или 
билеты.

%\begin{note*}
%В настоящем стандарте ссылочный билет передается от СИ к ПС через КП. 
%ПС напрямую обращается с ним к СИ и получает в ответ нужные билеты.
%\end{note*}

{\bf \thedefctr~статический пароль}:
Секрет аутентификации, который способен запомнить человек.

{\bf \thedefctr~сторона}:
Активный элемент: лицо, устройство, процесс, сервер, центр, служба.

{\bf \thedefctr~терминал}:
Сторона, которая представляет службу идентификации и организует по ее поручению
аутентификацию пользователей или самостоятельно проводит аутентификацию.

{\bf \thedefctr~токен аутентификации}; ТА:
Устройство или данные, которыми сторона владеет и которые использует для
аутентификации.

{\bf \thedefctr~удостоверение}:
Документ на физическом носителе, выпущенный доверенной стороной и содержащий
идентификационные данные пользователя.

% Физический документ, который содержит идентификационные данные  
% пользователя, возможно его биометрические характеристики.

{\bf \thedefctr~узел (OIDC)}:
Конечная коммуникационная точка сети прикладного уровня,
которая характеризуется интерфейсом, построенном на базе протокола HTTP,  
в том числе: уникальным сетевым именем, методом, принимаемыми параметрами,
возвращаемыми значениями.

% cетевое имя: URL path,  метод: HTTP method

{\bf \thedefctr~уровень гарантий аутентификации}:
Степень уверенности в том, что для пользователя, прошедшего аутентификацию
относительно некоторых утверждений, утверждения действительно выполняются.

% A category describing the strength of the authentication process

{\bf \thedefctr~уровень гарантий идентификации}:
Степень уверенности в том, что зарегистрированный и идентифицированный
пользователь действительно тот, за кого себя выдает.

% A category that conveys the degree of confidence that the applicant’s claimed 
% identity is their real identity

{\bf \thedefctr~уровень гарантий федерации}:
Степень защиты утверждений, распространяемых в федерации.

% Степень надежности отношений доверия в федерации?

% A category describing the assertion protocol used by the federation to communicate
% authentication and attribute information (if applicable) to an RP.

{\bf \thedefctr~утверждение}:
Характеристика стороны или события, в том числе данные о пользователе или
сведения о прохождении им аутентификации.

{\bf \thedefctr~утверждение аутентификации}:
Сведения об успешной аутентификации пользователя.

{\bf \thedefctr~фактор аутентификации}:
Одна из трех категорий токенов аутентификации как таковых
или данных, нужных для активации токенов:
1)~<<что я знаю>>, 2)~<<что я имею>>, 3)~<<кто я>>.

%\begin{note*}
%Примеры факторов: <<что я знаю>>~--- статический пароль, 
%<<что я имею>>~--- устройство, криптографический токен,
%<<кто я>>~--- биометрические характеристики, шаблон поведения.
%\end{note*}

{\bf \thedefctr~энтропия}:
Степень неопределенности. 

\begin{note*}
Если имеется $2^n$~вариантов выбора объекта и все эти варианты примерно
равновероятны, то говорят, что объект содержит $n$~битов энтропии.
\end{note*}

{\bf \thedefctr~федерация (доверия)}:
Совокупность сторон, связанных отношениями доверия, полного или частичного.

\begin{note*}
В настоящем стандарте речь идет о федерациях, которые используют 
сервисы инфраструктуры аутентификации, входят в состав инфрастуктуры или 
пересекаются с нею. 
%
Доверие в федерации основано на аутентификации сторон.
Доверие может транслироваться в авторизацию. 
\end{note*}

% A collection of realms (domains) that have established trust among themselves. 
% The level of trust may vary, but typically includes authentication and may 
% include authorization.

% A process that allows the conveyance of identity and authentication 
% information across a set of networked systems.

{\bf \thedefctr~центр федерации}; ЦФ:
Сторона, которая является гарантом отношений доверия в федерации.

% OIDC: federation operator, SP: federation authority

В настоящем стандарте используются следующие сокращения, дополнительные к 
введенным выше:

ИОК~--- инфраструктура открытых ключей;

СКЗИ~--- средство криптографической защиты информации;

ЭЦП~--- электронная цифровая подпись;

OIDC~--- OpenID Connect;

URI~--- Uniform Resource Identifier~\cite{RFC3986};

HTTP~--- Hypertext Transfer Protocol~\cite{RFC3986};

В приложении~\ref{ENG} для некоторых терминов настоящего стандарта 
представлены англоязычные прототипы, которые используются в 
спецификации OIDC~\cite{OIDC} и базовых для нее 
спецификациях~\cite{RFC6749,RFC6750}.

