% Session Management
\section{Управление сеансами (УC)}\label{SM}

\subsection{Обзор}\label{SM.Intro}

Пакет УС устанавливает требования по управлению аутентифицированными сеансами 
пользователя с ПС и СИ.

ПС открывает аутентифицированный сеанс после получения БА. Именно в этом сеансе 
пользователь выполняет операции, ПС оказывает ему цифровые услуги. Неформально, 
речь идет о работе пользователя в личном кабинете.
%
Сеанс с пользователем может открыть и СИ. В этом сеансе поддерживается статус
аутентификации пользователя, упрощается повторная аутентификация.

Сеанс строится по схеме <<клиент~--- сервер>>. В качестве клиента выступает КП
пользователя, в качестве сервера~--- СИ или ПС.

Связь в сеансе реализуется одним или несколькими соединениями. 
%
Соединение, как правило, устанавливается на транспортном 
сетевом уровне, а сеанс~--- на прикладном, поэтому в сеансе могут
быть неизвестны детали соединений.
%
Даже если к моменту создания сеанса уже имеются защищенные соединения,
в сеансе нельзя положиться на эту защиту.

Для организации защиты сеанса сервер выпускает БС и передает его КП. 
Предъявляя билет при последовательных обращениях к серверу,
КП подтверждает владение сеансом. Это подтверждение является основанием
для оказания пользователю цифровых услуг, для упрощения повторной 
аутентификации. БС позволяет пользователю поддержать длительный 
сеанс с сервером без постоянной аутентификации.  

Сеанс может завершиться по разным причинам. Во-первых, может просто истечь
срок его действия. Во-вторых, пользователь может прекратить сеанс по 
собственной инициативе, завершив работу с КП или выполнив логаут. В-третьих, 
сеанс может завершить сервер после длительного бездействия пользователя. 

Для продолжения сеанса проводится повторная аутентификация.
%
Она может быть проще первоначальной (см. требование~\ref{R.TI.AAL2}), 
может выполняться по другой схеме, например с использованием выпущенного 
ранее БА. 

\subsection{Требования}\label{SM.Reqs}

\req{УC}{1--3}
БС должен генерироваться сервером (ПС или СИ) сразу после аутентификации или 
предъявления БА. БС должен передаваться КП пользователя по защищенному 
соединению (с аутентификацией отправителя). 

\req{УC}{1--3}
КП не должна использовать БС вне защищенных соединений с его отправителем. 

\req{УC}{1--3}
БС должен содержать не менее 128 бит энтропии.

% todo: 64? См. ПА.1 и NIST SP800-63-3B

\req{УC}{1--3}
Сервер (ПС или СИ) должен хранить БС вместе со временем его выпуска.
Билет должен уничтожаться на стороне сервера по окончании срока действия.

\req{УС}{1--3}
КП пользователя не должна хранить БС в общедоступной области памяти.
Билет должен уничтожаться на стороне пользователя по окончании срока действия.

\begin{note*}
Для КП, выполненной в виде браузера, общедоступной является память,
которая может быть прочитана с помощью JavaScript, например,
локальное хранилище (local storage) HTML5.
%
% todo: Cookies:
% 1. SHALL be tagged to be accessible only on secure (HTTPS) sessions.
% 2. SHALL be accessible to the minimum practical set of hostnames and paths.
% 3. SHOULD be tagged to be inaccessible via JavaScript (HttpOnly).
% 4. SHOULD be tagged to expire at, or soon after, the session’s validity 
%    period. This requirement is intended to limit the accumulation of cookies, 
%    but SHALL NOT be depended upon to enforce session timeouts. 
\end{note*}

\req{УС}{1}
Срок действия БС не должен превышать 30~суток. После 12~ч бездействия 
пользователя должна выполняться повторная аутентификация.

\begin{note*}
Пользователь может предупреждаться о приближении повторной аутентификации.
\end{note*}

\req{УC}{2}
Срок действия БС не должен превышать 12~ч. После 30~мин бездействия 
пользователя должна выполняться повторная аутентификация. 

\req{УC}{3}
Срок действия БС не должен превышать 12~ч. После 15~мин бездействия 
пользователя должна выполняться повторная аутентификация.

