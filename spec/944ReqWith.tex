\hiddensection{Запрос утверждений}\label{CLAIMS.ReqWith}

\subsection{Запрос утверждений с помощью параметра \lstinline{scope}} 
\label{CLAIMS.ReqWithScope}

Утверждения о пользователе могут быть запрошены с помощью параметра
\lstinline{scope}. Параметр включается в запрос авторизации~/ аутентификации 
и учитывается СИ при выпуске БД и БА.
%
Утверждения возвращаются в ответе с узла UserInfo при предъявлении 
БД. 
%
Если БД не выпускается, то утверждения возвращаются в БА. 

Утверждения, которые запрашиваются с помощью параметра \lstinline{scope},
считаются несущественными.

Перечень запрашиваемых утверждений регулируется следующими лексемами 
\lstinline{scope}: 
\begin{itemize}
\item
\lstinline{"profile"}~--- запрашиваются утверждения профиля по умолчанию: 
\lstinline{name}, 
\lstinline{family_name}, 
\lstinline{given_name}, 
\lstinline{middle_name}, 
\lstinline{nickname},
\lstinline{preferred_username}, 
\lstinline{profile}, 
\lstinline{picture}, 
\lstinline{website}, 
\lstinline{gender}, 
\lstinline{birthdate},
\lstinline{zoneinfo}, 
\lstinline{locale}, 
\lstinline{updated_at};

\item
\lstinline{"email"}~--- 
запрашиваются утверждения
\lstinline{email} и \lstinline{email_verified};

\item
\lstinline{"address"}~--- 
запрашиваются утверждения 
\lstinline{address};

\item
\lstinline{"phone"}~--- 
запрашиваются утверждения
\lstinline{phone_number} и \lstinline{phone_number_verified}.
\end{itemize}

Пример запроса утверждений с помощью параметра \lstinline{scope}:
\begin{lstlisting}
scope=openid profile email phone
\end{lstlisting}

\subsection{Запрос утверждений с помощью параметра \lstinline{claims}}
\label{CLAIMS.ReqWithClaims}

Утверждения могут быть запрошены с помощью параметра \lstinline{claims}. 
Параметр включается в запрос авторизации~/ аутентификации и учитывается СИ при 
выпуске БД и БА.
%
Утверждения возвращаются в ответе с узла UserInfo при предъявлении БД и (или) в 
БА. 
%
Параметр \lstinline{claims} определяет перечень запрашиваемых утверждений, 
способ их возврата. 
%
В \lstinline{claims} предусмотрена детализация запроса по каждому утверждению.

Запрос \lstinline{claims} представляется объектом JSON, который включает два 
вложенных объекта: \lstinline{userinfo} и \lstinline{id_token}.
%
В объекте \lstinline{userinfo} перечисляются утверждения, которые следует возвратить 
с узла UserInfo, в объекте \lstinline{id_token}~--- утверждения, которые 
следует возвратить в БА.
%
Оба объекта являются необязательными.

Запрос \lstinline{claims} может включать другие объекты и, соответственно, 
перечни утверждений. Например, дополнительный перечень может касаться 
дополнительного узла ресурсов. Объекты \lstinline{claims}, которые не могут 
быть распознаны, должны игнорироваться.

Если в \lstinline{claims} включен объект \lstinline{userinfo}, то запрашиваемые 
в нем утверждения добавляются к перечню утверждений, запрашиваемых через 
параметр \lstinline{scope}. Включение \lstinline{userinfo} в \lstinline{claims}
должно сопровождаться добавлением лексемы \lstinline{"token"} в параметр
\lstinline{response_type}, т.е. запросом БД.

Если в \lstinline{claims} включен объект \lstinline{id_token}, то запрашиваемые 
в нем утверждения добавляются к перечню утверждений об аутентификации, 
которые по умолчанию (без явного запроса) включаются в БА. 

В объектах \lstinline{userinfo} и \lstinline{id_token} имена полей совпадают
с именами запрашиваемых утверждений. Поле либо принимает значение 
\lstinline{null}, либо значением поля является объект JSON с деталями запроса 
утверждения. 

Значение \lstinline{null} означает, что утверждение запрашивается
в режиме по умолчанию, в частности, как несущественное. 

\subsection{Детали запроса утверждения в \lstinline{claims}}\label{CLAIMS.IndClaim}

Объект JSON с деталями запроса утверждения включает поля 
\lstinline{essential}, \lstinline{value} и \lstinline{values},
каждое из которых является необязательным.
%
Могут вводиться дополнительные поля.  
%
Поле, которое не может быть распознано, должно игнорироваться.

Поле \lstinline{essential} содержит признак существенности утверждения. 
Например, утверждение \lstinline{auth_time} запрашивается как 
существенное следующим образом: 
\begin{lstlisting}
"auth_time": {"essential": true}
\end{lstlisting}

По умолчанию \lstinline{essential} принимает значение \lstinline{false}.

Поле \lstinline{value} задает определенное значение запрашиваемого 
утверждения. Например, сослаться на пользователя с идентификатором 
\lstinline{"248289761001"} при запросе утверждений можно следующим образом:
\begin{lstlisting}
"sub": {"value": "248289761001"}
\end{lstlisting}

В \lstinline{value} должно указываться корректное значение утверждения. 

Поле \lstinline{values} задает набор определенных значений запрашиваемого 
утверждения. Значения задаются в порядке убывания приоритета.
%
Например, запросить второй и первый уровни гарантии аутентификации,
отдавая приоритет второму, можно следующим образом:
\begin{lstlisting}
"acr": {"essential": true,
        "values": ["2", "1"]}
\end{lstlisting}

В \lstinline{values} должны указываться корректные значения утверждения.

Пример запроса утверждений через \lstinline{claims}:
\begin{lstlisting}
{
  "userinfo":
  {
    "given_name": {"essential": true},
    "nickname": null,
    "email": {"essential": true},
    "email_verified": {"essential": true},
    "picture": null
  },
  "id_token":
  {
    "auth_time": {"essential": true},
    "acr": {"values": ["2", "1"]}
   }
}
\end{lstlisting}

\subsection{Запрос утверждения \lstinline{acr}}\label{CLAIMS.ReqAcr}

Запрашивая утверждение \lstinline{acr} в параметре \lstinline{claims}, ПС 
предлагает СИ поддержать один из уровней гарантий пользователя, указанных в 
качестве допустимых значений \lstinline{acr}. 

Если утверждение \lstinline{acr} запрашивается как существенное,
то СИ должна поддержать предложение ПС. 
%
Для этого СИ может провести повторную аутентификацию пользователя. 
%
Если СИ не может поддержать один из запрошенных уровней гарантий, 
то это должно трактоваться как неуспешная попытка аутентификации.

Если утверждение \lstinline{acr} запрашивается как несущественное,
то СИ не обязательно поддерживать один из запрошенных уровней.
СИ может возвратить в \lstinline{acr} достигнутый уровень 
или вообще не возвращать \lstinline{acr}.

Передача предпочтительных уровней гарантий аутентификации в параметре 
\lstinline{acr_values} (см.~\ref{PARAMS.AcrValues}) также считается 
запросом утверждения \lstinline{acr}. При этом утверждение запрашивается
как несущественное. Поведение при запросе \lstinline{acr} одновременно через 
\lstinline{claims} и \lstinline{acr_values} не определено.
	