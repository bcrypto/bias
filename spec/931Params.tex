\hiddensection{Параметры}\label{PARAMS}

\subsection{Общие сведения}\label{PARAMS.Syntax}

В настоящем подразделе перечисляются параметры запросов и ответов OIDC, 
устанавливаются правила формирования, передачи и обработки параметров, 
определяются форматы параметров.
%
Параметры перечисляются в алфавитном порядке.

При определении форматов параметров используется синтаксис ABNF (Augmented 
Backus-Naur Form), установленный~\cite{RFC5234}. 
%
Применяются следующие определения ABNF, дополнительные к введенным~\cite{RFC5234}:
\begin{lstlisting}
VSCHAR = %x20-7E
NQCHAR = %x21 / %x23-5B / %x5D-7E
NQSCHAR = %x20-21 / %x23-5B / %x5D-7E
UNICODECHARNOCRLF = %x09 / %x20-7E / %x80-D7FF / %xE000-FFFD 
    / %x10000-10FFFF
base64url-char = ALPHA / DIGIT / "_" / "-"
base64url = *base64url-char
jws = base64url 2( "." base64url )
    ; header.payload.signature
jwe = base64url 4( "." base64url )
    ; header.encrypted-key.iv.ciphertext.tag
jwt = jws / jwe
\end{lstlisting}

Дополнительно используется определение \lstinline{URI-reference}, 
установленное в~\cite{RFC3986}.

% VSCHAR = all printable ASCII characters
% NQCHAR = VSCHAR без " ", """, "\"
% NQSCHAR = VSCHAR без """, "\"
% UNICODECHARNOCRLF definition is based upon the Char definition
%   in Section 2.2 of [W3C.REC-xml-20081126], but omitting the Carriage
%   Return and Linefeed characters.
% rfc5802:
%   base64-char     = ALPHA / DIGIT / "/" / "+"
%   base64-4        = 4base64-char
%   base64-3        = 3base64-char "="
%   base64-2        = 2base64-char "=="
%   base64          = *base64-4 [base64-3 / base64-2]
% Base64url Encoding
%   Base64 encoding using the URL- and filename-safe character set
%   defined in Section 5 of RFC 4648 [RFC4648], with all trailing '='
%   characters omitted (as permitted by Section 3.2) and without the
%   inclusion of any line breaks, whitespace, or other additional
%   characters.  Note that the base64url encoding of the empty octet
%   sequence is the empty string.  (See Appendix C for notes on
%   implementing base64url encoding without padding.)
% In the JWS Compact Serialization, a JWS is represented as the
% concatenation:
%   BASE64URL(UTF8(JWS Protected Header)) || '.' ||
%   BASE64URL(JWS Payload) || '.' ||
%   BASE64URL(JWS Signature)
% In the JWE Compact Serialization, a JWE is represented as the
% concatenation:
%   BASE64URL(UTF8(JWE Protected Header)) || '.' ||
%   BASE64URL(JWE Encrypted Key) || '.' ||
%   BASE64URL(JWE Initialization Vector) || '.' ||
%   BASE64URL(JWE Ciphertext) || '.' ||
%   BASE64URL(JWE Authentication Tag)

Параметрами ответа с узла UserInfo являются утверждения \lstinline{sub}, 
\lstinline{iss}, \lstinline{aud}. Они описываются в~\ref{CLAIMS.Auth}.

\subsection{Параметр \lstinline{access_token}}\label{PARAMS.AccessToken}

Параметр \lstinline{access_token} содержит БД, выпущенный СИ.

Параметр \lstinline{access_token} возвращается в ответах с узлов Authorization и 
Token, передается в запросах на узел UserInfo и другие узлы ресурсов.
%
Параметр должен включаться в успешный ответ с узла 
Authorization, если и только если параметр \lstinline{response_type} запроса 
на узел содержит строку \lstinline{"token"} 
(см. таблицу~\ref{Table.OIDC.RespType}).

БД должен содержать не менее 128 битов энтропии.

Формат параметра:
\begin{lstlisting}
access-token = 22*VSCHAR
\end{lstlisting}

%todo: https://stackoverflow.com/questions/50031993/what-characters-are-allowed-in-an-oauth2-access-token

\subsection{Параметр \lstinline{acr_values}}\label{PARAMS.AcrValues} 

Параметр \lstinline{acr_values} содержит перечень предпочтительных
уровней гарантий аутентификации пользователя. Уровни кодируются лексемами, 
разделяемыми пробелами. Уровни перечисляются в порядке убывания предпочтения.

Параметр \lstinline{acr_values} указывается в запросе на узел Authorization.

СИ должна учитывать указанные в \lstinline{acr_values} предпочтения при выборе 
методов аутентификации пользователя.
% 
СИ может указать выбранный уровень в утверждении \lstinline{acr} БА 
(см. \ref{CLAIMS.Acr}).
%
% OIDC: The acr Claim is requested as a Voluntary Claim by this parameter.

\begin{note*}
Предпочтительные уровни гарантий аутентификации могут запрашиваться также через 
параметр \lstinline{claims} (см. \ref{PARAMS.Claims}, \ref{CLAIMS.ReqAcr}).
\end{note*}

Формат параметра:
\begin{lstlisting}
acr-values = 1*VSCHAR
\end{lstlisting}

Базовый уровень гарантий рекомендуется кодировать строкой \lstinline{"1"},
средний~--- строкой \lstinline{"2"}, высокий~--- \lstinline{"3"}.
%
Могут вводиться подуровни: \lstinline{"11"} (1-й подуровень базового уровня), 
\lstinline{"12"} (2-й подуровень) и т.д.

\subsection{Параметр \lstinline{claims}}\label{PARAMS.Claims} 

Параметр \lstinline{claims} содержит перечень утверждений,
которые требуется включить в БА и/или ответы с узлов ресурсов.
%
Перечень представляет собой объект JSON, в котором указываются 
имена утверждений, их ожидаемые значения (например, идентификатор пользователя) 
и атрибуты (например, признак обязательности включения в билет).
%
Правила формирования перечня определены в~\ref{CLAIMS.ReqWith}.

Параметр \lstinline{claims} указывается в запросе на узел Authorization.

Формат параметра:
\begin{lstlisting}
claims = 1*VSCHAR
\end{lstlisting}

\subsection{Параметр \lstinline{claims_locales}}\label{PARAMS.ClaimsLocales} 

Параметр \lstinline{claims_locales} содержит перечень предпочтительных 
языков утверждений.
%
Перечень состоит из тегов, формируемых по правилам~\cite{RFC5646}. 
%
Тег кодирует язык, а также возможно алфавит (кириллица или латиница), 
региональные особенности, диалект и пр. 
%
Теги разделяются пробелами и перечисляются в порядке убывания предпочтения.

Параметр \lstinline{claims_locales} указывается в запросе на узел Authorization.

СИ и СР должны выбирать язык утверждений с учетом перечисленных в 
\lstinline{claims_locales} тегов и их очередности. 
%
При обработке запроса СИ не следует возвращать ошибку, даже если ни один из 
запрашиваемых языков не поддерживается.

Формат параметра:
\begin{lstlisting}
claims-locales = locales-name *( SP locales-name )
locales-name = 1*locales-char
locales-char = "-" / DIGIT / ALPHA
\end{lstlisting}

\subsection{Параметр \lstinline{client_assertion}}
\label{PARAMS.ClientAssertion}

Параметр \lstinline{client_assertion} представляет собой объект JWT, 
который выступает в роли аутентификатора ПС при обращении к СИ.
%
Объект содержит волатильные контекстнозависимые данные, которые сопровождаются 
имитовставкой или подписью ПС. 

Параметр \lstinline{client_assertion} указывается в запросе на узел Token.
%
Параметр должен включаться в запрос при использовании методов аутентификации 
\lstinline{client_secret_jwt} и \lstinline{private_key_jwt}.

Формат параметра:
\begin{lstlisting}
client-assertion = jwt
\end{lstlisting}

Исходными данными передаваемого в \lstinline{client_assertion} объекта JWT 
являются утверждения. Обязательные утверждения:
\lstinline{iss}, \lstinline{sub}, \lstinline{aud}, \lstinline{jti}, 
\lstinline{exp}, 
необязательное: \lstinline{iat} (см. \ref{CLAIMS.Auth}).
%
Могут использоваться другие утверждения. Утверждения, которые СИ не может 
распознать, должны игнорироваться.

В утверждениях \lstinline{iss} и \lstinline{sub} должен быть указан
идентификатор ПС, в утверждении \lstinline{aud}~--- идентификатор СИ.
В качестве идентификатора СИ может выступать адрес узла Token. При обработке 
объекта СИ должна проверять, что утверждение \lstinline{aud} действительно 
относится к ней.

В утверждении \lstinline{jti} ПС должна указать уникальный идентификатор 
объекта JWT. СИ может проверять уникальность, сохраняя предъявляемые объекты 
вплоть до окончания их действия.

Момент окончания действия должен указываться в утверждении \lstinline{jti}. 
СИ должна отклонять просроченные объекты (с поправкой на возможную 
рассинхронизацию). 
%
СИ может отклонять объекты, продолжительность действия которых неоправданно 
велика.

В утверждении \lstinline{iat} ПС может указать момент выпуска объекта. 
%
СИ может отклонять объекты, которые начали действовать неоправданно 
давно.

\subsection{Параметр \lstinline{client_assertion_type}}
\label{PARAMS.ClientAssertionType} 

Параметр \lstinline{client_assertion_type} описывает метод аутентификации ПС,
который требует передачи аутентификатора в параметре 
\lstinline{client_assertion}.

Параметр \lstinline{client_assertion_type} указывается в запросе на узел Token.
%
Параметр должен включаться в запрос при использовании методов аутентификации 
\lstinline{client_secret_jwt} и \lstinline{private_key_jwt}.

Формат параметра:
\begin{lstlisting}
client-assertion-type = 1*( "_" / "-" / ":" / DIGIT / ALPHA )
\end{lstlisting}

При использовании методов аутентификации \lstinline{client_secret_jwt} и 
\lstinline{private_key_jwt} параметр должен принимать значение 
\begin{lstlisting}
"urn:ietf:params:oauth:client-assertion-type:jwt-bearer"
\end{lstlisting}

\subsection{Параметр \lstinline{client_id}}\label{PARAMS.ClientId}

Параметр \lstinline{client_id} содержит идентификатор ПС. Идентификатор 
назначает СИ при регистрации ПС. 

Параметр \lstinline{client_id} указывается в запросах на узлы Authorization и 
Token. 
%
Параметр должен включаться в запрос на узел Token, если аутентификация 
выполняется в момент обращения к узлу и для этого используется  
метод \lstinline{client_secret_basic} или \lstinline{client_secret_post}.

Формат параметра:
\begin{lstlisting}
client-id = *VSCHAR
\end{lstlisting}

% todo: Пустой client-id?

\subsection{Параметр \lstinline{client_secret}}\label{PARAMS.ClientSecret}

Параметр \lstinline{client_secret} содержит зарегистрированный секрет 
аутентификации ПС. 

Параметр \lstinline{client_secret} указывается в запросе на узел Token.
%
Параметр должен включаться в запрос, если аутентификация 
выполняется в момент обращения к узлу и для этого используется  
метод \lstinline{client_secret_basic} или \lstinline{client_secret_post}.

Формат параметра:
\begin{lstlisting}
client-secret = 20*VSCHAR
\end{lstlisting}

\subsection{Параметр \lstinline{code}}\label{PARAMS.Code}

Параметр \lstinline{code} содержит код авторизации, выпущенный СИ
по результатам обработки запроса авторизации / аутентификации в схеме
Code или Hybrid.

Параметр \lstinline{code} возвращается в ответе с узла Authorization, 
передается в запросе на узел Token.

При выпуске кода авторизации СИ должна связать его с идентификатором ПС 
и адресом перенаправления (параметры \lstinline{client_id} и 
\lstinline{redirect_uri} запроса на узел Authorization) и учитывать 
связывание при приеме кода.
%
Код авторизации должен содержать не менее 64 битов энтропии.
%
Для снижения издержек при компрометации кода срок его действия должен быть 
небольшим.
%
Рекомендуемый срок: не более 10~мин.

ПС должна использовать код авторизации однократно. При повторном получении 
одного и того же кода СИ не должна его обрабатывать и, более того, СИ следует   
отменить все билеты, выпущенные ранее на основании этого кода.

Формат параметра:
\begin{lstlisting}
code = 10*VSCHAR
\end{lstlisting}

\subsection{Параметр \lstinline{display}}\label{PARAMS.Display}  

Параметр \lstinline{display} определяет предпочтительный тип интерфейса 
пользователя во время аутентификации перед СИ и дачи разрешения на доступ к его 
ресурсам. Интерфейс реализует КП.

Параметр \lstinline{display} указывается в запросе на узел Authorization.

При обработке запроса СИ может определить возможности клиентской программы и 
самостоятельно выбрать подходящий тип интерфейса.

Формат параметра:
\begin{lstlisting}
display = 1*( "_" / DIGIT / ALPHA )
\end{lstlisting}

Стандартные значения параметра:
%
\begin{itemize}
\item
\lstinline{"page"}~--- полностраничный режим, значение по умолчанию;

\item
\lstinline{"popup"}~--- 
всплывающее окно. Окно должно иметь подходящий размер и не должно 
закрывать главное окно КП;

\item
\lstinline{"touch"}~---
интерфейс сенсорного устройства;

\item
\lstinline{"wap"}~---
интерфейс сотового телефона (не смартфона).
\end{itemize}

\subsection{Параметр \lstinline{error}}\label{PARAMS.Error}

Параметр \lstinline{error} описывает код ошибки.

Параметр \lstinline{error} включается в ответ об ошибке.
%
Параметр следует включать в ответ с узла UserInfo, если ошибка касается 
технологии OIDC и не может быть описана стандартным кодом ошибки HTTP.

Формат параметра:
\begin{lstlisting}
error = 1*NQSCHAR
\end{lstlisting}

Стандартные коды ошибок:
\begin{itemize}
\item
% OAuth[Auth,Token,UserInfo]
\lstinline{"invalid_request"}~---
неверный формат запроса: отсутствует требуемый параметр, некорректное значение
параметра, параметр встречается более одного раза, другое.
%
Код \lstinline{"invalid_request"} должен возвращаться вместе с кодом 400 
(Bad Request) HTTP;

\item
% OAuth[Token]
\lstinline{"invalid_client"}~---
ошибка аутентификации ПС: неизвестная ПС, аутентификация не выполнялась, 
неподдерживаемый метод аутентификации, другое.
%
Код \lstinline{"invalid_client"} может возвращаться вместе с кодом 401 
(Unauthorized) HTTP.
%
Если секрет аутентификации передавался в заголовке \lstinline{Authorization} 
HTTP-запроса (например, использовался метод \lstinline{client_secret_basic}),
то код 401 является обязательным и в HTTP-ответ должен быть включен заголовок 
\lstinline{WWW-Authenticate} с информацией об использованном методе 
аутентификации;
%
% todo: использованном? ("matching the auth. scheme _used_ by the client")

\item
% OAuth[Token]
\lstinline{"invalid_grant"}~---
код авторизации, аутентификатор пользователя или БО недействителен: 
некорректен, уже не действуют, отозван, 
не соответствует адресу перенаправления, выпущен для другой ПС;
%
% todo: аутентификатор пользователя vs "resource owner credentials"

\item
% OAuth[UserInfo]
\lstinline{"invalid_token"}~--- 
БД недействителен: некорректен, уже не действует, отозван, другое. 
%
Код \lstinline{"invalid_token"} должен возвращаться вместе с кодом 
401 (Unauthorized) HTTP;

\item
% OAuth[UserInfo]
\lstinline{"insufficient_scope"}~---
область действия БД недостаточна для успешной обработки запроса.
%
Код \lstinline{"insufficient_scope"} следует возвращать с кодом 
403 (Forbidden) HTTP. 
%
Код \lstinline{"insufficient_scope"} может сопровождаться
параметром \lstinline{scope}, в котором должен быть указан требуемая
область действия;

\item
% OAuth[Auth,Token]
\lstinline{"unauthorized_client"}~--- 
ПС не авторизована на обращение к узлу, запрос кода авторизации или БД;

\item
% OAuth[Auth]
\lstinline{"access_denied"}~---
отказ в обработке запроса со стороны пользователя или СИ;

\item
% OAuth[Auth]
\lstinline{"unsupported_response_type"}~---
СИ не поддерживает заявленное значение параметра \lstinline{response_type}; 

\item
% OAuth[Token]
\lstinline{"unsupported_grant_type"}~---
СИ не поддерживает заявленное значение параметра \lstinline{grant_type};

\item
% OAuth[Auth,Token]
\lstinline{"invalid_scope"}~---
указанное в параметре \lstinline{scope} значение некорректно, не распознано, 
или выходит за пределы заданной пользователем области действия;

\item
% OAuth[Auth]
\lstinline{"server_error"}~---
СИ столкнулась с непредвиденными обстоятельствами, 
которые препятствуют обработке запроса;

\item
% OAuth[Auth]
\lstinline{"temporarily_unavailable"}~---
СИ в настоящее время неспособна обработать запрос из-за временной перегрузки 
или технического обслуживания;

\begin{note*}
Коды \lstinline{"server_error"}, \lstinline{"temporarily_unavailable"} 
имеют такое же значение, что и коды 500 (Internal Server Error), 
503 (Service Unavailable) HTTP. Однако коды HTTP не могут быть возвращены  
ПС через механизм перенаправления.
\end{note*}

\item
% OIDC[Auth]
\lstinline{"interaction_required"}~---
СИ требует взаимодействия с пользователем в той или иной форме. 
%
Код \lstinline{"interaction_required"} может возвращаться тогда, когда параметр 
\lstinline{prompt} запроса авторизации~/ аутентификации принимает значение 
\lstinline{"none"}, и при этом запрос не может быть обработан без отображения 
интерфейса пользователя для взаимодействия с ним; 

\item
% OIDC[Auth]
\lstinline{"login_required"}~---
СИ требует аутентификации пользователя.
%
Код~\lstinline{"login_required"} может возвращаться тогда,
когда параметр \lstinline{prompt} запроса авторизации~/ аутентификации 
принимает значение \lstinline{"none"}, и при этом запрос не может быть 
обработан без отображения интерфейса пользователя для его аутентификации; 

\item
% OIDC[Auth]
\lstinline{"account_selection_required"}~--- 
для аутентификации пользователь должен выбрать учетную запись. 
%
Пользователь может быть аутентифицирован перед СИ с использованием различных 
учетных записей, но не выбрана ни одна из них.
%
Код~\lstinline{"account_selection_required"} может возвращаться тогда, когда 
параметр \lstinline{prompt} запроса авторизации~/ аутентификации 
принимает значение \lstinline{"none"}, и при этом запрос не может быть 
обработан без отображения интерфейса пользователя для выбора учетной записи;

\item
% OIDC[Auth]
\lstinline{"consent_required"}~--- 
СИ требует подтверждения согласия на доступ к ресурсам пользователя.
%
Код~\lstinline{"consent_required"} может возвращаться тогда, когда 
параметр \lstinline{prompt} запроса авторизации~/ аутентификации 
принимает значение \lstinline{"none"}, и при этом запрос не 
может быть обработан без отображения интерфейса пользователя для подтверждения 
его согласия на доступ к ресурсам; 

% OIDC[Auth]
\item
\lstinline{"invalid_request_uri"}~---
параметр \lstinline{request_uri} в запросе авторизации~/ аутентификации 
ссылается на некорректный объект JWT или ошибка при обращении  
по указанному в \lstinline{request_uri} адресу; 

% OIDC[Auth]
\item
\lstinline{"invalid_request_object"}~--- 
параметр \lstinline{request} в запросах авторизации~/ аутентификации содержит 
некорректный объект JWT; 

% OIDC[Auth]
\item
\lstinline{"request_not_supported"}~---
СИ не поддерживает параметр \lstinline{request} в запросах авторизации~/ 
аутентификации;

% OIDC[Auth]
\item
\lstinline{"request_uri_not_supported"}~---
СИ не поддерживает параметр \lstinline{request_uri} в запросах 
авторизации~/ аутентификации.
\end{itemize}

% todo: OIDC, Section 7
% OIDC[Auth]
% \lstinline{"registration_not_supported"}~---
% СИ не поддерживает параметр \lstinline{registration} в запросах 
% авторизации~/ аутентификации.

\subsection{Параметр \lstinline{error_description}}\label{PARAMS.ErrorDescr}

Параметр \lstinline{error_description} содержит информацию об ошибке в удобной 
для восприятия человеком форме.

Параметр \lstinline{error_description} включается в ответ об ошибке.

% todo: error_description без error?

Формат параметра:
\begin{lstlisting}
error-description = 1*NQSCHAR
\end{lstlisting}

\subsection{Параметр \lstinline{error_uri}}\label{PARAMS.ErrorUri}

Параметр \lstinline{error_uri} содержит сетевой адрес (URI) веб-страницы 
с информацией об ошибке в удобной для восприятия человеком форме.

Параметр \lstinline{error_uri} включается в ответ об ошибке.

% todo: error_uri без error?

Формат параметра:
\begin{lstlisting}
error-uri = URI-reference
\end{lstlisting}

\subsection{Параметр \lstinline{expires_in}}\label{PARAMS.ExpiresIn}

Параметр \lstinline{expires_in} определяет срок действия БД в секундах. 
%
Например, значение 300 означает, что БД действует в течение 5~мин с момента 
формирования ответа.

Параметр \lstinline{expires_in} включается в ответы с узлов Authorization и 
Token. 
%
Параметр рекомендуется включать в ответ, если ответ содержит БД.  
%
Если параметр опущен, то СИ должна определить в документации или другим способом
срок действия БД, используемый по умолчанию.

Формат параметра:
\begin{lstlisting}
expires-in = 1*DIGIT
\end{lstlisting}

\subsection{Параметр \lstinline{grant_type}}\label{PARAMS.GrantType}

Параметр \lstinline{grant_type} определяет тип запроса на узел Token.

Параметр \lstinline{grant_type} включается в запрос на узел Token.
%
В запросе на выпуск билетов параметр должен принимать значение 
\lstinline{"authorization_code"}.
%
В запросе на обновление билетов параметр должен принимать значение 
\lstinline{"refresh_token"}.

Формат параметра:
\begin{lstlisting}
grant-type = grant-name / URI-reference
grant-name = 1*name-char
name-char = "-" / "." / "_" / DIGIT / ALPHA
\end{lstlisting}

\subsection{Параметр \lstinline{id_token}}\label{PARAMS.IdToken}

Параметр \lstinline{id_token} содержит БА, выпущенный СИ.

Параметр \lstinline{id_token} возвращается в ответах с узлов Authorization и Token.

Формат параметра: 
\begin{lstlisting}
id-token = 1*VSCHAR
\end{lstlisting}

Содержание и формат БА детализируются в приложении~\ref{IDTOKEN}. 

% todo: возврат БА при обновлении билетов (в ответ на БО)

\subsection{Параметр \lstinline{id_token_hint}}\label{PARAMS.IdTokenHint}

Параметр \lstinline{id_token_hint} содержит БА, ранее выпущенный СИ.
БА подтверждает, что пользователь был аутентифицирован в одном из предыдущих 
сеансов со СИ.

Параметр \lstinline{id_token_hint} указывается в запросе на узел Authorization. 

СИ возвращает успешный ответ на запрос, если сеанс с пользователем, указанным в 
БА, не завершен или если сеанс был открыт в результате обработки запроса. В 
противном случае СИ следует возвратить ответ об ошибке, например, с кодом 
\lstinline{"login_required"}.

Если параметр \lstinline{prompt} запроса на узел Authorization 
принимает значение \lstinline{"none"}, то параметр \lstinline{id_token_hint} 
следует включать в запрос. 
%
Если \lstinline{id_token_hint} все-таки не включен, то СИ может возвратить 
ошибку.
%
Тем не менее, СИ следует давать по возможности успешные ответы даже при 
отсутствии в запросе \lstinline{id_token_hint}.

БА, переданный через \lstinline{id_token_hint}, может использоваться СИ, 
даже если идентификатор СИ не указан в утверждении \lstinline{aud} билета 
(см. \ref{CLAIMS.Aud}).

Если БА, который требуется передать через \lstinline{id_token_hint}, 
не только подписан, но и конвертован, то ПС перед передачей должна снять защиту 
с БА, а затем конвертовать билет на открытом ключе СИ.

Формат параметра:
\begin{lstlisting}
id-token-hint = 1*VSCHAR
\end{lstlisting}

\subsection{Параметр \lstinline{login_hint}}\label{PARAMS.LoginHint}

Параметр \lstinline{login_hint} содержит идентификатор, который пользователь 
предположительно может использовать во время аутентификации перед СИ. 
%
Предполагаемый идентификатор может быть известен ПС, если в сеансе с 
пользователем тот передавал ей свои идентификационные данные.
%
Передача предполагаемого идентификатора может упростить СИ поиск учетной записи 
пользователя.

Параметр \lstinline{login_hint} указывается в запросе на узел Authorization.

Обработка параметра \lstinline{login_hint} остается на усмотрение СИ.
% todo: It is RECOMMENDED that the hint value match the value used for 
% discovery. 

Формат параметра:
\begin{lstlisting}
login-hint = *UNICODECHARNOCRLF
\end{lstlisting}

\subsection{Параметр \lstinline{max_age}}\label{PARAMS.MaxAge} 

Параметр \lstinline{max_age} описывает срок действия утверждений
аутентификации, т.е. максимальный промежуток времени с момента (активной)
аутентификации пользователя, в течение которого ПС готова полагаться на факт
аутентификации.
%
Срок действия задается в секундах.

Параметр \lstinline{max_age} указывается в запросе на узел Authorization.

Если заданный в \lstinline{max_age} срок действия истек (например, задано 
нулевое значение), то СИ должна инициировать аутентификацию пользователя.

Если параметр \lstinline{max_age} включен в запрос, то выпускаемый БА должен 
содержать утверждение \lstinline{auth_time} (см. \ref{CLAIMS.AuthTime}).

Формат параметра:
\begin{lstlisting}
max-age = 1*DIGIT
\end{lstlisting}

\subsection{Параметр \lstinline{nonce}}\label{PARAMS.Nonce} 

Параметр \lstinline{nonce} содержит синхропосылку~--- 
волатильные данные, предназначенные для противодействия атакам на основе 
повторов. Синхропосылка повторяется в запросе аутентификации и соответствующем 
БА, делая их уникальными и связывая друг с другом. 

Параметр \lstinline{nonce} указывается в запросе на узел Authorization.
%
Параметр должен включаться в запрос при использовании коммуникационных 
схем Implicit и Hybrid.

Синхропосылку готовит ПС. Синхропосылка должна содержать не менее 64 битов 
энтропии.

При обработке запроса СИ должна сохранить синхропосылку, переданную в параметре
\lstinline{nonce}, а затем перенести ее в выпускаемый БА. Синхропосылка 
повторяется в утверждении \lstinline{nonce} билета.

Если синхропосылка включена в БА, то при получении билета ПС должна сравнить ее 
с синхропосылкой, отправленной в запросе.

Формат параметра:
\begin{lstlisting}
nonce = 10*VSCHAR
\end{lstlisting}

\subsection{Параметр \lstinline{prompt}}\label{PARAMS.Prompt}

Параметр \lstinline{prompt} содержит перечень приглашений пользователю для
организации его повторной аутентификации и получения его согласия на доступ к
ресурсам. Приглашения кодируются лексемами, разделяемыми пробелами.

Параметр \lstinline{prompt} указывается в запросе на узел Authorization.

ПС может включать параметр в запрос для проверки того, что сеанс пользователя 
все еще активен или для привлечения внимания пользователя к запросу.

% todo: The prompt parameter can be used by the Client to make sure that the 
% End-User is still present for the current session or to bring attention to 
% the request.  

Формат параметра:
\begin{lstlisting}
prompt = prompt-name *( SP prompt-name )
prompt-name = 1*prompt-char
prompt-char = "_" / DIGIT / ALPHA
\end{lstlisting}

Предусмотрены следующие приглашения:
%
\begin{itemize}
\item
\lstinline{"none"}~--- 
СИ не должна отображать элементы интерфейса пользователя и, таким образом, 
приглашение фактически отсутствует, являясь техническим.
%
Если пользователь еще не аутентифицирован или ПС не имеет предварительного
согласия на доступ к ресурсам пользователя или не выполнены какие-либо другие 
условия, то должна быть возвращена ошибка, например, с кодом 
\lstinline{"login_required"} или \lstinline{"interaction_required"}.
%
Приглашение \lstinline{"none"} может использоваться для проверки того, 
аутентифицирован пользователь или нет;

% todo: "or does not fulfill other conditions for processing the request"?
 
\item
\lstinline{"login"}~--- 
СИ следует предложить пользователю пройти повторную аутентификацию.
%
Если СИ не может повторно аутентифицировать пользователя, то должна быть 
возвращена ошибка, например, с кодом \lstinline{"login_required"};

\item
\lstinline{"consent"}~--- 
СИ следует предложить пользователю подтвердить согласие на доступ ПС к его 
ресурсам. 
%
Если СИ не может получить подтверждение, то должна быть возвращена
ошибка, например, с кодом \lstinline{"consent_required"};

\item
\lstinline{"select_account"}~--- 
СИ следует предложить пользователю выбрать учетную запись.
%
Предложение оказывается полезным, если у пользователя имеется несколько учетных 
записей и нужно определить подходящую для текущего сеанса.
%
Если СИ не может получить информацию о сделанном пользователем выборе 
учетной записи, то должна быть возвращена ошибка, например, с кодом
\lstinline{"account_selection_required"}.
\end{itemize}

Приглашение \lstinline{"none"} не должно включаться в \lstinline{prompt} с 
какими-либо другими приглашениями. 

Если параметр \lstinline{scope} запроса на узел Authorization включает значение
\lstinline{"offline_access"}, то после обработки запроса ПС может иметь
доступ к ресурсам пользователя, даже если тот не активен (офлайн).
%
Такой запрос должен содержать параметр \lstinline{prompt} со значением 
\lstinline{"consent"}, которое обязует СИ запросить разрешение пользователя 
на доступ к его ресурсам. 
%
Разрешение должно запрашиваться всякий раз при передаче \lstinline{scope}
со значением \lstinline{"offline_access"}, предыдущие разрешения могут 
оказаться недостаточными.
%
Исключением из этого правила являются ситуации, когда пользователь дает 
разрешение на доступ другим способом, например, во время регистрации.

\subsection{Параметр \lstinline{redirect_uri}}\label{PARAMS.RedirectUri}

Параметр \lstinline{redirect_uri} содержит адрес узла Redirection.
Адрес используется при перенаправлении ПС ответов СИ через КП пользователя.

Параметр \lstinline{redirect_uri} указывается в запросах на узлы Authorization 
и Token.

ПС должен указать в запросе на узел Authorization один из адресов перенаправления, 
согласованных со СИ во время регистрации. СИ должна проверить, что передан 
зарегистрированный адрес. Адреса должны сравниваться как строки, посимвольно.

ПС должен указать в запросе на узел Token тот же адрес, что был передан ранее 
в запросе на узел Authorization. СИ должна проверить совпадение адресов.

Формат параметра:
\begin{lstlisting}
redirect-uri = URI-reference
\end{lstlisting}

% todo: When using this [Code] flow, the Redirection URI SHOULD use the https 
% scheme; however, it MAY use the http scheme, provided that the Client Type is 
% confidential, as defined in Section 2.1 of OAuth 2.0, and provided the OP 
% allows the use of http Redirection URIs in this case. The Redirection URI MAY 
% use an alternate scheme, such as one that is intended to identify a callback 
% into a native application. 

% todo: When using this [Implicit, Hybrid] flow, the Redirection URI MUST NOT 
% use the http scheme unless the Client is a native application, in which case 
% it MAY use the http: scheme with localhost as the hostname.

\subsection{Параметр \lstinline{refresh_token}}\label{PARAMS.RefreshToken} 

Параметр \lstinline{refresh_token} содержит БО, выпущенный СИ.
БО используется при обновлении билетов, в том числе самого БО.

Параметр \lstinline{refresh_token} возвращается в ответе с узла Token и 
передается в запросе на этот узел.
%
БО должен включаться в успешный ответ, если параметр \lstinline{scope} 
запроса авторизации~/ аутентификации содержит значение 
\lstinline{"offline_access"}. 

БО должен содержать не менее 128 битов энтропии.

При обновлении БО область действия, заданная параметром \lstinline{scope}, 
должна сохраняться.
%
Получив обновленный БО, ПС должна использовать его вместо текущего.

Формат параметра:
\begin{lstlisting}
refresh-token = 22*VSCHAR
\end{lstlisting}

\subsection{Параметр \lstinline{request}}\label{PARAMS.Request} 

Параметр \lstinline{request} содержит объект JWT, в котором в качестве утверждений
представлены параметры запроса. Объект можно подписывать и (или) 
конвертовать и, таким образом, организовать защиту параметров запроса.

Параметр \lstinline{request} указывается в запросе на узел Authorization.

Параметры \lstinline{request} и \lstinline{request_uri} не должны включаться в 
запрос одновременно. 
%
Объект JWT, передаваемый в \lstinline{request}, не должен включать 
параметры \lstinline{request} и \lstinline{request_uri}.

В объекте JWT следует передавать долгосрочные параметры (которые могут быть 
предварительно подписаны), в том время как волатильные параметры (например,
\lstinline{nonce} или \lstinline{state}) следует передавать обычным образом. 
%
Параметры объекта JWT имеют приоритет перед параметрами, переданными обычным 
образом.

Параметры \lstinline{response_type} и \lstinline{client_id} должны передаваться
обычным образом. При включении параметров в объект JWT их значения должны
повторяться.

Параметр \lstinline{scope} также должен передаваться обычным образом и при 
этом содержать строку \lstinline{"openid"}, которая указывает на использование 
технологии OIDC.

Если объект JWT подписывается, то в него следует включать утверждения 
\lstinline{iss} и \lstinline{aud} (см.~\ref{CLAIMS.Iss}, \ref{CLAIMS.Aud}).
%
В утверждении \lstinline{iss} следует указать идентификатор ПС или другой 
стороны, подписавшей объект.
%
В утверждении \lstinline{aud} следует указать идентификатор СИ.

Объект JWT может не только подписываться, но и конвертоваться на открытом ключе 
СИ после подписания или конвертоваться без подписания. Конвертование перед 
подписанием запрещено.

Формат параметра:
\begin{lstlisting}
request = jwt
\end{lstlisting}

\subsection{Параметр \lstinline{request_uri}}\label{PARAMS.RequestUri} 

Параметр \lstinline{request_uri} содержит сетевой адрес объекта JWT с параметрами 
запроса. Содержание объекта и правила работы с ним определены при описании
параметра \lstinline{request}. Отличие только в том, что в \lstinline{request}
объект передается по значению, а в \lstinline{request_uri}~--- по ссылке.
%
Передача по ссылке может быть удобнее, если объект имеет большой размер.

Параметр \lstinline{request_uri} указывается в запросе на узел Authorization.

Параметры \lstinline{request} и \lstinline{request_uri} не должны включаться в 
запрос одновременно. 

Формат параметра:
\begin{lstlisting}
request-uri = URI-reference
\end{lstlisting}

Если объект JWT не подписан или если подпись не может проверена СИ, 
то в адресе объекта JWT должна использоваться схема \lstinline{https}.
%
СИ должна иметь доступ к объекту по указанному адресу.
%
Следует также предусмотреть доступ ПС.

\subsection{Параметр \lstinline{response_mode}}\label{PARAMS.RespMode} 

Параметр \lstinline{response_mode} содержит предпочтительный режим ответа СИ на 
запрос авторизации~/ аутентификации.

Параметр \lstinline{response_mode} указывается в запросе на узел Authorization.

Параметр не рекомендуется использовать в тех случаях, когда запрашиваемый 
режим ответа является режимом по умолчанию в соответствии с типом, переданным
в \lstinline{response_type}. 
%
% todo: This use of this parameter is NOT RECOMMENDED when the Response Mode 
% that would be requested is the default mode specified for the Response Type.

Формат параметра:
\begin{lstlisting}
response-mode = 1*( "_" / DIGIT / ALPHA )
\end{lstlisting}

Предусмотрены следующие режимы ответа:
%
\begin{itemize}
\item
\lstinline{"query"}~--- 
параметры ответа кодируются по схеме Query (см. \ref{OIDC.Prelim});

\item
\lstinline{"fragment"}~--- 
параметры ответа кодируются по схеме Fragment.
\end{itemize}

\subsection{Параметр \lstinline{response_type}}\label{PARAMS.RespType}

Параметр \lstinline{response_type} определяет тип ответа на запрос 
авторизации~/ аутентификации и неявно задает коммуникационную схему, которая 
будет использоваться. 

Параметр \lstinline{response_mode} указывается в запросе на узел Authorization.

Формат параметра:
\begin{lstlisting}
response-type = response-name *( SP response-name )
response-name = 1*response-char
response-char = "_" / DIGIT / ALPHA
\end{lstlisting}

Допустимые значения параметра определены в таблице~\ref{Table.OIDC.RespType}.

\subsection{Параметр \lstinline{scope}}\label{PARAMS.Scope}

Параметр \lstinline{scope} определяет область действия запроса авторизации,
в том числе используемую технологию (OIDC), перечень выпускаемых билетов,
права доступа к ресурсам пользователей. Область кодируется лексемами, 
разделяемыми пробелами. С увеличением числа лексем область становится шире.

Параметр \lstinline{scope} указывается в запросах на узлы Authorization и 
Token, возвращается в ответах с узлов Authorization, Token и UserInfo.
%
В запросах параметр определяет запрашиваемую область действия, в успешных 
ответах~--- разрешенную (согласованную) область действия, в ответах об ошибке 
(с узла UserInfo)~--- область действия, которая требуется для получения 
доступа. 

Параметр \lstinline{scope} должен включаться в успешный ответ, если 
согласованная область действия отличается от запрошенной. Если области 
совпадают, то включение не обязательно.

Область действия в запросе на обновление билетов не должна быть шире, чем 
область действия в запросе авторизации~/ аутентификации. Наоборот, область
может сужаться. 
%
Если область действия не указывается в запросе на обновление билетов,
то СИ должна полагать, что повторяется область соответствующего запроса
авторизации~/ аутентификации.

Формат параметра:
\begin{lstlisting}
scope = scope-token *( SP scope-token )
scope-token = 1*NQCHAR
\end{lstlisting}

Область действия должна содержать лексему \lstinline{"openid"},
указывающую на использование технологии OIDC. 

Область действия может включать лексему \lstinline{"offline_access"}, 
если от СИ требуется возврат БО. 

Область действия может содержать дополнительные лексемы, которые 
определяют запрашиваемые ПС в виде утверждений ресурсы пользователя
(см.~\ref{CLAIMS.ReqWithScope}). 

Лексемы \lstinline{scope}, которые не могут быть распознаны, должны 
игнорироваться.

\subsection{Параметр \lstinline{state}}\label{PARAMS.State} 

Параметр \lstinline{state} содержит состояние ПС~--- волатильные данные, 
связывающие запросы авторизации~/ аутентификации с соответствующими ответами. 
%
Связывание блокирует атаки типа CSRF (см.~\ref{ATK.TM}).

Параметр \lstinline{state} указывается в запросе на узел Authorization и 
возвращается в соответствующих ответах.
%
Состояние должно включаться в ответ (успешный и об ошибке), если оно указано в 
запросе. 

Состояние из запроса должно повторяться в ответе. При получении ответа ПС 
должен проверять данный факт.

Формат параметра:
\begin{lstlisting}
state = 1*VSCHAR
\end{lstlisting}

% https://developers.google.com/identity/protocols/oauth2/openid-connect?hl=en#createxsrftoken
% https://stackoverflow.com/questions/26132066/what-is-the-purpose-of-the-state-parameter-in-oauth-authorization-request
	
\subsection{Параметр \lstinline{token_type}}\label{PARAMS.TokenType} 

Параметр \lstinline{token_type} определяет тип БД, выпущенного СИ.

Параметр \lstinline{token_type} указывает в ответе с узла Token.
%
Параметр должен включаться в ответ, если ответ содержит параметр 
\lstinline{access_token}. 

Формат параметра:
\begin{lstlisting}
token-type = type-name / URI-reference
type-name = 1*name-char
name-char = "-" / "." / "_" / DIGIT / ALPHA
\end{lstlisting}

Параметр должен принимать значение~\lstinline{"Bearer"}. 
%
Использование других типов БД выходит за пределы настоящего стандарта. 

\subsection{Параметр \lstinline{ui_locales}}\label{PARAMS.UiLocales}

Параметр \lstinline{ui_locales} содержит перечень предпочтительных 
языков интерфейса пользователя.
%
Перечень состоит из тегов, формируемых по правилам~\cite{RFC5646}. 
%
Теги разделяются пробелами и перечисляются в порядке убывания предпочтения.

Параметр \lstinline{ui_locales} указывается в запросе на узел Authorization.

СИ должна выбирать язык утверждений с учетом перечисленных в 
\lstinline{ui_locales} тегов и их очередности. 
%
При обработке запроса СИ не следует возвращать ошибку, даже если ни один из 
запрашиваемых языков не поддерживается.

Формат параметра:
\begin{lstlisting}
ui-locales = locales-name *( SP locales-name )
locales-name = 1*locales-char
locales-char = "-" / DIGIT / ALPHA
\end{lstlisting}

